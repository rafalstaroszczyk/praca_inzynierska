\documentclass[10pt, a4paper, twoside, onecolumn]{article}
\usepackage[backend=biber, sorting=none]{biblatex}
\addbibresource{refs.bib}
\usepackage[utf8]{inputenc}
\usepackage[T1]{fontenc}
\usepackage[polish]{babel}
\usepackage{uarial}
\usepackage[top=2.5cm,bottom=2.5cm,inner=3.5cm,outer=2.5cm]{geometry}
\usepackage{fancyhdr}
\usepackage{indentfirst}
\usepackage{graphicx}
\usepackage{hyperref}
\usepackage{float}
\usepackage{multirow}
\usepackage{amsmath}
\usepackage{sectsty}
\usepackage{etoolbox}
\usepackage[font=small]{caption}

%\usepackage{titling}
\usepackage{anyfontsize}
\usepackage{blindtext}

\graphicspath{ {./} }
\setlength{\parindent}{1.25cm}
\setlength{\parskip}{12pt}

\pagestyle{fancy}
\fancyhf{}
\fancyfoot[C]{\fontfamily{ua1}\fontsize{9pt}{9pt}\selectfont\thepage}
\renewcommand{\headrulewidth}{0pt}
\renewcommand{\footrulewidth}{0pt}

\renewcommand{\familydefault}{ua1}
\renewcommand{\baselinestretch}{1.5}

\setcounter{page}{3}
\raggedbottom

\sectionfont{\noindent\fontsize{12}{15}\selectfont}
\subsectionfont{\noindent\fontsize{10}{15}\selectfont\textit}
\subsubsectionfont{\noindent\fontsize{10}{15}\selectfont\normalfont\textit}

\BeforeBeginEnvironment{tabular}{\small}

\numberwithin{equation}{section}

% Problemy:
% W spisie kropki przy section
% Odpowiednie przerwy miedzy akapitami
% Opisy tabel nad tabela
% Odpowiednie przerwy miedzy opisem a tabela/rysunkiem
% Byc moze problemy z numerowaniem tabel, ale chyba jest na razie dobrze

\begin{document}
	\section*{Streszczenie}
	\begin{center}
		\textbf{Strona 3}
	\end{center}
	\blindtext \par
	\blindtext \par
	\blindtext \par
	\blindtext \par
	\pagebreak
	
	\section*{Abstract}
	\begin{center}
		\textbf{Strona 4}
	\end{center}
	\blindtext \par
	\blindtext \par
	\blindtext \par
	\blindtext \par
	\pagebreak
	
	\section*{Spis treści}
	\begin{center}
		\textbf{Strona 5}
	\end{center}
	\tableofcontents
	\pagebreak
	
	\section*{Wykaz oznaczeń}
	\addcontentsline{toc}{section}{Wykaz oznaczeń}
	\setlength{\parindent}{0cm}
	\begin{table}[H]
	\begin{tabular}{@{} ll}
		\(T\) & Temperatura \\
		\(V\) & Objętość \\
		\(p\) & Ciśnienie \\
		\(E\) & Energia \\
		\(S\) & Entropia \\
		\(W=E+pV\) & Entalpia \\
		\(F=E-TS\) & Energia swobodna \\
		\(\Phi=E-TS+pV\) & Potencjał termodynamiczny (entalpia swobodna) \\
		\(\Omega=-pV\) & Wielki potencjał termodynamiczny \\
		\(C_{p}, C_{V}\) & Pojemności cieplne (\(c_{p}, c_{V}\) ciepła własciwe) \\
		\(N\) & Liczba cząsteczek (cząstek) \\
		\(\mu\) & Potencjał chemiczny \\
		\(\alpha\) & Współczynnik napięcia powierzchniowego \\
		\(s\) & Pole powierzchni rozdziału faz
	\end{tabular}
	\end{table}
	\setlength{\parindent}{1.25cm}
	\pagebreak
	
	\section{Wstęp}
	\subsection{Definicje i prawa}
	
	\subsection{Zarys historyczny}
	Termodynamika równowagowa zajmuje się procesami, w których ignoruje się upływ czasu, a przemiana jest kwazistatyczna. Oznacza to, że każdy stan pośredni można traktować jako stan równowagi termodynamicznej. Model taki jest wystarczający do opisu większości procesów. Można więc powiedzieć, że termodynamikę równowagową interesuje stan początkowy oraz końcowy. \par
	Termodynamika nierównowagowa jednak zajmuje się dokładnie tym, co się dzieje w trakcie rzeczywistej przemiany i jest ona konieczna do opisu reakcji oscylacyjnych. Pierwsze przesłanki o istnieniu takowych sięgają końca XIX wieku. Były to reakcje w układach heterogenicznych, jak na przykład pierścienie Lieseganga lub oscylacje prądu płynącego przez ogniwo galwaniczne. Wyjaśnienie tych zjawisk wymagało, aby układ byl heterogeniczny i było w zgodzie z entropią Boltzmanna, według której spontaniczna organizacja jest niemożliwa. \cite{orlik}\par
	Pierwszy model teoretyczny został przedstawiony przez Alfreda Lotka \cite{lotka}. Przez długi czas uważano, że nie mogą one przedstawiać rzeczywistych reakcji, ponieważ łamią II Z.T. według Boltzmanna. Jednak w 1921r. pokazano w reakcji Bray'a-Liebhafky'ego, że reakcje oscylacyjne w układach homogenicznych są możliwę. Jest to reakcja rokładu nadtlenku wodoru katalizowana jodanem (V). Jeszcze większy wpływ na rozwój termodynamiki nierównowagowej w kinetyce chemicznej były reakcje Biełousowa-Żabotyńskiego. Pierwszą reakcją z tej grupy została zaobserwowanaw 1959 w mieszaninie bromianu (V) potasu, siarczanie (VI) ceru (IV), kwasu malonowego oraz kwasu cytrynowego w rozcieńczonym kwasie siarkowym (VI). Została ona odkryta jako nieorganiczny analog cyklu Krebsa \cite{belousov_hist}. Istnienie takich reakcji jest jednak niezgodne z oryginalną definicją entropii Boltzmanna. 
	
	\subsubsection{Podpodsekcja 1.1.1}
	\blindtext
	\section{Sekcja 2}
	\blindtext
	\section{Sekcja 3}
	\blindtext
	\begin{table}
	\centering
		\begin{tabular}{|cc|}
			1 & 2 \\
			3 & 4 
		\end{tabular}
		\caption{Tabela 1}
		\label{table:1}
	\end{table}
	\section{Sekcja 4}
	\blindtext 
	\pagebreak
	%\section*{Wykaz literatury}
	\addcontentsline{toc}{section}{Wykaz literatury}
	\printbibliography[title=Wykaz literatury]
	\pagebreak
	\section*{Wykaz rysunków}
	\addcontentsline{toc}{section}{Wykaz rysunków}
	\pagebreak
	\section*{Wykaz tabel}
	\addcontentsline{toc}{section}{Wykaz tabel}
	\pagebreak
	\section*{Dodatek A}
	\addcontentsline{toc}{section}{Dodatek A}
	\pagebreak
\end{document}