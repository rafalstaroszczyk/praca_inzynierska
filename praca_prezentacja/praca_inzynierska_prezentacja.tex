\documentclass{beamer}
\usetheme{Warsaw}

\usepackage[utf8]{inputenc}
\usepackage[T1]{fontenc}
\usepackage[polish]{babel}
\usepackage{amsmath}
\usepackage{amssymb}
\usepackage{physics}
\usepackage[version=4]{mhchem}
\usepackage[font=small]{caption}
\usepackage{subfig}

\graphicspath{ {../praca/} }

\title{Opis wybranych układów fizykochemicznych w stanie nierównowagi termodynamicznej}

\author{Rafał Staroszczyk\\promotor: dr Piotr Weber}
\institute{Politechnika Gdańska}
\date{\today}

\begin{document}

\begin{frame}
\titlepage
\end{frame}

\begin{frame}
\frametitle{Spis treści}
\tableofcontents
\end{frame}

\begin{frame}
\frametitle{Cel pracy}
\begin{itemize}
\item Przedstawienie teorii termodynamiki nierównowagowej, kinetyki reakcji chemicznych oraz modeli reakcji oscylacyjnych
\item Analiza teoretyczna modeli reakcji chemicznych oscylacyjnych
\item Symulacja numeryczna tych modeli
\end{itemize}
\end{frame}

\begin{frame}
\frametitle{Model Lotki}
\begin{columns}
\column{0.5\textwidth}
Historycznie pierwszy model, który można zastosować do reakcji chemicznych oscylacyjnych. Oryginalnie użyty do modelowania populacji. \\
\begin{figure}
\includegraphics[width=\textwidth]{"lotka3; a=0.1".png}
\end{figure}
\column{0.5\textwidth}
\begin{block}{Równania reakcji}
\begin{center}
	\ce{A ->[k_1] X} \\
	\ce{X + Y ->[k_2] 2Y} \\
	\ce{Y ->[k_3] produkty}
\end{center}
\end{block}
\begin{block}{Równania różniczkowe}
\begin{gather*}
	\dv{x}{\tau} = a-axy \\
	\dv{y}{\tau} = xy-y
\end{gather*}
\end{block}
\end{columns}
\end{frame}

\begin{frame}
\frametitle{Model Lotki-Volterry}
\begin{columns}
\column{0.5\textwidth}
Modyfikacja modelu Lotki z autokatalizą. Wykres fazowy jest zawsze torem zamkniętym.  
\begin{figure}
\includegraphics[width=\textwidth]{"lotka_volterra; a=1".png}
\end{figure}
\column{0.5\textwidth}
\begin{block}{Równania reakcji}
\begin{center}
	\ce{A + X ->[k_1] 2X} \\
	\ce{X + Y ->[k_2] 2Y} \\
	\ce{Y ->[k_3] produkty}
\end{center}
\end{block}
\begin{block}{Równania różniczkowe}
\begin{gather*}
	\dv{x}{\tau} = ax-axy \\
	\dv{y}{\tau} = xy-y
\end{gather*}
\end{block}
\end{columns}
\end{frame}

\begin{frame}
\frametitle{Model bruskelator}
\begin{columns}
\column{0.5\textwidth}

\begin{figure}
\includegraphics[width=\textwidth]{"brusselator; a=7; b=4".png}
\end{figure}
\column{0.5\textwidth}
\begin{block}{Równania reakcji}
\begin{center}
	\ce{A ->[k_1] X} \\
	\ce{2X + Y ->[k_2] 3X} \\
	\ce{B + X ->[k_3] D + Y} \\
	\ce{X ->[k_4] E}
\end{center}
\end{block}
\begin{block}{Równania różniczkowe}
\begin{gather*}
	\dv{x}{\tau} = 1+ax^{2}y-ax-x \\
	\dv{y}{\tau} = -bx^{2}y+bx
\end{gather*}
\end{block}
\end{columns}
\end{frame}

\begin{frame}
\frametitle{Model bruskelator ogólny}
\begin{columns}
\column{0.5\textwidth}
Uogólniona wersja modelu bruskelator, w których każda z reakcji jest odwracalna. 
\begin{figure}
\includegraphics[width=\textwidth]{"brusselator_rev1; a=9, b=1, c=1, d=0.1".png}
\end{figure}
\column{0.5\textwidth}
\begin{block}{Równania reakcji}
\begin{center}
	\ce{A <=>[k_{1}][k_{-1}] X} \\
	\ce{2X + Y <=>[k_{2}][k_{-2}] 3X} \\
	\ce{B + X <=>[k_{3}][k_{-3}] D + Y} \\
	\ce{X <=>[k_{4}][k_{-4}] E}
\end{center}
\end{block}
\begin{block}{Równania różniczkowe}
\begin{gather*}
	\dv{x}{\tau} = 1 + ax^{2}y - ax - x \\
	- cx - bc^{3} + by + c \\
	\dv{y}{\tau} = -bx^{2}y + bx + dx^{3} - dy
\end{gather*}
\end{block}
\end{columns}
\end{frame}

\begin{frame}
\frametitle{Produkcja entropii}
\begin{block}{Źródło produkcji entropii}
\begin{equation*}
\sigma=\vb{J}_{U}\vdot\grad{\left(\frac{1}{T}\right)}-\sum_{i}\vb{J}_{i}\cdot\grad{\left(\frac{\mu_{i}}{T}\right)}+\sum_{r}\frac{A_{r}}{T}\dv{\xi_{r}}{t}
\end{equation*}
\end{block}
\end{frame}

\begin{frame}
\begin{columns}
\column{0.5\textwidth}

\column{0.5\textwidth}
\begin{figure}
\subfloat[Reakcja Biełousowa-Żabotyńskiego]{\includegraphics[width=0.45\textwidth]{"belousov_zhabotinsky_2".png}}
\subfloat[Komórki B\'{e}narda]{\includegraphics[width=0.45\textwidth]{"benard".png}}
\end{figure}
\end{columns}
\end{frame}

\subsection{Reakcje chemiczne}

\section{Wyniki}


\end{document}