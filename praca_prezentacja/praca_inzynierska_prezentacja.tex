\documentclass{beamer}
\usetheme{Warsaw}

\usepackage[backend=biber, sorting=none, maxbibnames=999]{biblatex}
\addbibresource{../praca/refs.bib}
\usepackage[utf8]{inputenc}
\usepackage[T1]{fontenc}
\usepackage[polish]{babel}
\usepackage{amsmath}
\usepackage{amssymb}
\usepackage{physics}
\usepackage[version=4]{mhchem}
\usepackage[font=small]{caption}
\usepackage{subfig}

\graphicspath{ {../praca/} }

\title{Opis wybranych układów fizykochemicznych w stanie nierównowagi termodynamicznej}

\author{Rafał Staroszczyk\\promotor: dr Piotr Weber}
\institute{Politechnika Gdańska}
\date{\today}

\setbeamertemplate{headline}{}

\setbeamertemplate{footline}[frame number]{}
\setbeamertemplate{navigation symbols}{}
\setbeamertemplate{footline}{}

\begin{document}
\beamertemplatenavigationsymbolsempty
\addtobeamertemplate{block begin}{\setlength\abovedisplayskip{0pt}}

\begin{frame}
\titlepage
\end{frame}

\begin{frame}
\frametitle{Spis treści}
\tableofcontents
\end{frame}

\begin{frame}
\frametitle{Cel pracy}
\begin{itemize}
\item Przedstawienie teorii termodynamiki nierównowagowej, kinetyki reakcji chemicznych oraz modeli reakcji oscylacyjnych
\item Analiza teoretyczna modeli reakcji chemicznych oscylacyjnych
\item Symulacja numeryczna tych modeli
\end{itemize}
\end{frame}

\begin{frame}
\begin{columns}
\column{0.5\textwidth}

\column{0.5\textwidth}
\begin{figure}
\subfloat[Reakcja Biełousowa-Żabotyńskiego]{\includegraphics[width=0.45\textwidth]{"belousov_zhabotinsky_2".png}}
\subfloat[Komórki B\'{e}narda]{\includegraphics[width=0.45\textwidth]{"benard".png}}
\end{figure}
\end{columns}
\end{frame}

\section{Teoria}
\subsection{Produkcja entropii}

\begin{frame}
\frametitle{Produkcja entropii}
\begin{columns}
\column{0.5\textwidth}
\begin{block}{}
\begin{gather*}
\dv{S}{t}=\frac{1}{T}\dv{U}{t}-\sum_{i}\frac{\mu_{i}}{T}\dv{n_{i}}{t} \\
\vb{J}_{S}=\frac{1}{T}\vb{J}_{U}-\sum_{i}\frac{\mu_{i}}{T}\vb{J}_{i} \\
A_{r}=-\sum_{i}\nu_{ir}\mu_{i}
\end{gather*}
\end{block}
\column{0.5\textwidth}
\begin{block}{Równania ciągłości}
\begin{gather*}
	\dv{U}{t}=-\div{\vb{J}_{U}} \\
	\dv{S}{t} = -\div{\vb{J}_{S}} + \sigma \\
	\dv{n_{i}}{t}=-\div{\vb{J}_{i}}+\dv{n_{i;reak}}{t}
\end{gather*}
\end{block}
\end{columns}
\begin{block}{Źródło produkcji entropii}
\begin{equation*}
\sigma=\vb{J}_{U}\vdot\grad{\left(\frac{1}{T}\right)}-\sum_{i}\vb{J}_{i}\cdot\grad{\left(\frac{\mu_{i}}{T}\right)}+\sum_{r}\frac{A_{r}}{T}\dv{\xi_{r}}{t}\geq0
\end{equation*}
\end{block}
\end{frame}

\subsection{Mechanizm i szybkość reakcji chemicznej}

\begin{frame}
\frametitle{Mechanizm reakcji chemicznej}
\begin{columns}
\column{0.5\textwidth}
Szybkość reakcji chemicznej jest w ogólności dowolną funkcją wszystkich stężeń. Mechanizm reakcji opisuje poszczególne reakcje elementarne.
\begin{block}{Równania różniczkowe}
\begin{gather*}
%\dv{\left[\ce{H2}\right]}{t} = -v_{2} \\
%\dv{\left[\ce{Br2}\right]}{t} = - v_{1} - v_{3} \\
%\dv{\left[\ce{H^{.}}\right]}{t} = v_{2} - v_{3} \\
%\dv{\left[\ce{Br^{.}}\right]}{t} = 2v_{1} - v_{2} + v_{3} \\
%\dv{\left[\ce{HBr}\right]}{t} = v_{2} + v_{3}
\dv{c}{t} = \nu v
\end{gather*}
\end{block}
\column{0.5\textwidth}
\begin{block}{Mechanizm reakcji}
\begin{center}
\ce{Br2 -> 2Br^{.}} \\
\ce{Br^{.} + H2 -> HBr + H^{.}} \\
\ce{H^{.} + Br2 -> HBr + Br^{.}}
\end{center}
\end{block}
\begin{block}{Szybkość reakcji}
\begin{gather*}
v_{1} = k_{1}\left[\ce{Br2}\right] \\
v_{2} = k_{2}\left[\ce{Br^{.}}\right]\left[\ce{H2}\right] \\
v_{3} = k_{3}\left[\ce{H^{.}}\right]\left[\ce{Br2}\right]
\end{gather*}
\end{block}
\end{columns}
\end{frame}

\section{Modele reakcji chemicznych oscylacyjnych}
\subsection{Model Lotki}

\begin{frame}
\frametitle{Model Lotki}
\begin{columns}
\column{0.5\textwidth}
Historycznie pierwszy model, który można zastosować do reakcji chemicznych oscylacyjnych. Oryginalnie użyty do modelowania populacji. \\
\begin{figure}
\includegraphics[width=\textwidth]{"lotka3; a=0.1".png}
\end{figure}
\column{0.5\textwidth}
\begin{block}{Równania reakcji}
\begin{center}
	\ce{A ->[k_1] X} \\
	\ce{X + Y ->[k_2] 2Y} \\
	\ce{Y ->[k_3] produkty}
\end{center}
\end{block}
\begin{block}{Równania różniczkowe}
\begin{gather*}
	\dv{x}{\tau} = a-axy \\
	\dv{y}{\tau} = xy-y
\end{gather*}
\end{block}
\end{columns}
\end{frame}

\subsection{Model Lotki-Volterry}

\begin{frame}
\frametitle{Model Lotki-Volterry}
\begin{columns}
\column{0.5\textwidth}
Modyfikacja modelu Lotki z autokatalizą. Wykres fazowy jest zawsze torem zamkniętym.  
\begin{figure}
\includegraphics[width=\textwidth]{"lotka_volterra; a=1".png}
\end{figure}
\column{0.5\textwidth}
\begin{block}{Równania reakcji}
\begin{center}
	\ce{A + X ->[k_1] 2X} \\
	\ce{X + Y ->[k_2] 2Y} \\
	\ce{Y ->[k_3] produkty}
\end{center}
\end{block}
\begin{block}{Równania różniczkowe}
\begin{gather*}
	\dv{x}{\tau} = ax-axy \\
	\dv{y}{\tau} = xy-y
\end{gather*}
\end{block}
\end{columns}
\end{frame}

\subsection{Model bruskelator}

\begin{frame}
\frametitle{Model bruskelator}
\begin{columns}
\column{0.5\textwidth}
Model ten osiąga taki sam cykl graniczny niezależnie od stężeń początkowych. 
\begin{figure}
\includegraphics[width=\textwidth]{"brusselator; a=7; b=4".png}
\end{figure}
\column{0.5\textwidth}
\begin{block}{Równania reakcji}
\begin{center}
	\ce{A ->[k_1] X} \\
	\ce{2X + Y ->[k_2] 3X} \\
	\ce{B + X ->[k_3] D + Y} \\
	\ce{X ->[k_4] E}
\end{center}
\end{block}
\begin{block}{Równania różniczkowe}
\begin{gather*}
	\dv{x}{\tau} = 1+ax^{2}y-ax-x \\
	\dv{y}{\tau} = -bx^{2}y+bx
\end{gather*}
\end{block}
\end{columns}
\end{frame}

\subsection{Model bruskelator ogólny}

\begin{frame}
\frametitle{Model bruskelator ogólny}
\begin{columns}
\column{0.5\textwidth}
Uogólniona wersja modelu bruskelator, w których każda z reakcji jest odwracalna. 
\begin{figure}
\includegraphics[width=\textwidth]{"brusselator_rev1; a=9, b=1, c=1, d=0.1".png}
\end{figure}
\column{0.5\textwidth}
\begin{block}{Równania reakcji}
\begin{center}
	\ce{A <=>[k_{1}][k_{-1}] X} \\
	\ce{2X + Y <=>[k_{2}][k_{-2}] 3X} \\
	\ce{B + X <=>[k_{3}][k_{-3}] D + Y} \\
	\ce{X <=>[k_{4}][k_{-4}] E}
\end{center}
\end{block}
\begin{block}{Równania różniczkowe}
\begin{gather*}
	\dv{x}{\tau} = 1 + ax^{2}y - ax - x \\
	- cx - bc^{3} + by + c \\
	\dv{y}{\tau} = -bx^{2}y + bx + dx^{3} - dy
\end{gather*}
\end{block}
\end{columns}
\end{frame}

\begin{frame}
\frametitle{Podsumowanie}

\end{frame}

\begin{frame}
\begin{center}
Dziękuję za uwagę
\end{center}
\end{frame}



\begin{frame}[allowframebreaks]
\frametitle{Literatura}
\nocite{*}
\printbibliography
\end{frame}

\end{document}