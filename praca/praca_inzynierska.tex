\documentclass[10pt, a4paper, twoside, onecolumn]{article}
\usepackage[backend=biber, sorting=none]{biblatex}
\addbibresource{refs.bib}
\usepackage[utf8]{inputenc}
\usepackage[T1]{fontenc}
\usepackage[polish]{babel}
\usepackage{uarial}
\usepackage[top=2.5cm,bottom=2.5cm,inner=3.5cm,outer=2.5cm]{geometry}
\usepackage{fancyhdr}
\usepackage{indentfirst}
\usepackage{graphicx}
\usepackage{hyperref}
\usepackage{float}
\usepackage{multirow}
\usepackage{amsmath}
\usepackage{sectsty}
\usepackage{etoolbox}
\usepackage[font=small]{caption}

%\usepackage{titling}
\usepackage{anyfontsize}
\usepackage{blindtext}

\graphicspath{ {./} }
\setlength{\parindent}{1.25cm}
\setlength{\parskip}{12pt}

\pagestyle{fancy}
\fancyhf{}
\fancyfoot[C]{\fontfamily{ua1}\fontsize{9pt}{9pt}\selectfont\thepage}
\renewcommand{\headrulewidth}{0pt}
\renewcommand{\footrulewidth}{0pt}

\renewcommand{\familydefault}{ua1}
\renewcommand{\baselinestretch}{1.5}

\setcounter{page}{3}
\raggedbottom

\sectionfont{\noindent\fontsize{12}{15}\selectfont}
\subsectionfont{\noindent\fontsize{10}{15}\selectfont\textit}
\subsubsectionfont{\noindent\fontsize{10}{15}\selectfont\normalfont\textit}

\BeforeBeginEnvironment{tabular}{\small}

\numberwithin{equation}{section}

\newcommand{\dbar}{d\hspace*{-0.08em}\bar{}\hspace*{0.1em}}

% Problemy:
% W spisie kropki przy section
% Odpowiednie przerwy miedzy akapitami
% Opisy tabel nad tabela
% Odpowiednie przerwy miedzy opisem a tabela/rysunkiem
% Byc moze problemy z numerowaniem tabel, ale chyba jest na razie dobrze

\begin{document}
	\section*{Streszczenie}
	\begin{center}
		\textbf{Strona 3}
	\end{center}
	%\blindtext \par
	%\blindtext \par
	%\blindtext \par
	%\blindtext \par
	\pagebreak
	
	\section*{Abstract}
	\begin{center}
		\textbf{Strona 4}
	\end{center}
	%\blindtext \par
	%\blindtext \par
	%\blindtext \par
	%\blindtext \par
	\pagebreak
	
	\section*{Spis treści}
	\begin{center}
		\textbf{Strona 5}
	\end{center}
	\tableofcontents
	\pagebreak
	
	\section*{Wykaz oznaczeń}
	\addcontentsline{toc}{section}{Wykaz oznaczeń}
	\setlength{\parindent}{0cm}
	\begin{table}[H]
	\begin{tabular}{@{} ll}
		\(T\) & Temperatura \\
		\(V\) & Objętość \\
		\(p\) & Ciśnienie \\
		\(E\) & Energia \\
		\(S\) & Entropia \\
		\(W=E+pV\) & Entalpia \\
		\(F=E-TS\) & Energia swobodna \\
		\(\Phi=E-TS+pV\) & Potencjał termodynamiczny (entalpia swobodna) \\
		\(\Omega=-pV\) & Wielki potencjał termodynamiczny \\
		\(C_{p}, C_{V}\) & Pojemności cieplne (\(c_{p}, c_{V}\) ciepła własciwe) \\
		\(N\) & Liczba cząsteczek (cząstek) \\
		\(\mu\) & Potencjał chemiczny \\
		\(\alpha\) & Współczynnik napięcia powierzchniowego \\
		\(s\) & Pole powierzchni rozdziału faz
	\end{tabular}
	\end{table}
	\setlength{\parindent}{1.25cm}
	\pagebreak
	
	\section{Wstęp}
	\subsection{Definicje i prawa}
	\begin{description}
		\item[Energia wewnętrzna] U; suma wszytkich energii kinetycznych oraz potencjalnych cząstek składających się na ciało
		\item[Praca] W; energia dostarczona do układu poprzez siły makroskopowe
		\item[Ciepło] Q; energia dostarczona do układu poprzez oddziaływania inne niż praca
		\item[Entropia (termodynamika klasyczna)] S; funkcja stanu układu; wielkość opisująca samorzutność przemian w systemie; wprowadzona po raz pierwszy przez Clausiusa jako \(dS=\frac{\dbar Q^\circ}{T}\) \cite{orlik}
		\item[Entropia (termodynamika statystyczna)] S; wielkość opisująca ilość mikrostanów przypisanych danemu makrostanowi; wprowadzona przez Boltzmanna: \(S=k_B\ln{\Omega}\) \cite{landau}
		\item[0 Zasada Termodynamiki] Jeżeli układy A i C są w równowadze termodynamicznej oraz B i C to A i B są w równowadze termodynamicznej
		\item[I Zasada Termodynamiki] Zwiększenie energii wewnętrznej może nastąpić poprzez pracę lub ciepło; \(dU=\dbar Q+\dbar W\)
		\item[II Zasada Termodynamiki] w układach izolowanych następuje wzrost entropii lub pozostaje ona stała (poza wyjątkiem fluktuacji); entropia podukładu może maleć \cite{landau}
		\item[III Zasada Termodynamiki] entropia kryształów doskonałych dąży do zera, gdy temperatura dąży do zera
	\end{description}
	\subsection{Zarys historyczny}
	Termodynamika równowagowa zajmuje się procesami, w których ignoruje się upływ czasu, a przemiana jest kwazistatyczna. Oznacza to, że każdy stan pośredni można traktować jako stan równowagi termodynamicznej. Model taki jest wystarczający do opisu większości procesów. Można więc powiedzieć, że termodynamikę równowagową interesuje stan początkowy oraz końcowy. \par
	Termodynamika nierównowagowa jednak zajmuje się dokładnie tym, co się dzieje w trakcie rzeczywistej przemiany i jest ona konieczna do opisu reakcji oscylacyjnych. Pierwsze przesłanki o istnieniu takowych sięgają końca XIX wieku. Były to reakcje w układach heterogenicznych, jak na przykład pierścienie Lieseganga lub oscylacje prądu płynącego przez ogniwo galwaniczne. Wyjaśnienie tych zjawisk wymagało, aby układ byl heterogeniczny i było w zgodzie z entropią Boltzmanna, według której spontaniczna organizacja jest niemożliwa. \cite{orlik}\par
	Pierwszy model teoretyczny został przedstawiony przez Alfreda Lotka \cite{lotka}. Przez długi czas uważano, że nie mogą one przedstawiać rzeczywistych reakcji, ponieważ łamią II Z.T. według Boltzmanna. Jednak w 1921r. pokazano w reakcji Bray'a-Liebhafky'ego, że reakcje oscylacyjne w układach homogenicznych są możliwę. Jest to reakcja rokładu nadtlenku wodoru katalizowana jodanem (V). Jeszcze większy wpływ na rozwój termodynamiki nierównowagowej w kinetyce chemicznej były reakcje Biełousowa-Żabotyńskiego. Pierwszą reakcją z tej grupy została zaobserwowanaw 1959 w mieszaninie bromianu (V) potasu, siarczanie (VI) ceru (IV), kwasu malonowego oraz kwasu cytrynowego w rozcieńczonym kwasie siarkowym (VI). Została ona odkryta jako nieorganiczny analog cyklu Krebsa \cite{belousov_hist}. Istnienie takich reakcji jest jednak niezgodne z oryginalną definicją entropii Boltzmanna. \par
	Innym zjawiskiem wyjaśnionym dzięki rozwoju termodynamiki nierównowagowej jest życie i ewolucja. Niepoprawne użycie II Z.T. może doprowadzić do wniosku, że powstanie złożonej struktury z chaosu powinno być niemożliwe. Wyjaśnienie takie błędnie wykorzystuje to prawo ignorując fakt, że układy biologiczne jak i cała Ziemia nie są układami izolowanymi. \par
	
	\subsection{Generacja entropii}
	II Zasada Termodynamiki odnosi się do układów zamkniętych.
	\begin{equation}\label{entropia_izolowany}
		dS_{uk}\geq 0
	\end{equation}
	Wynika z niego, że w układach izolowanych entropia zawsze rośnie, a zatem spontaniczne uporządkowanie stabilnych struktur nie jest możliwe. \par
	Zmianę entropii układu można zapisać jako wynikającą z przepływu ciepła oraz samorzutnej generacji entropii: 
	\begin{equation}
		dS_{uk}=d_{e}S+d_{i}S_{uk}
	\end{equation}
	Dla układu zamkniętego: \(d_{e}S=0\), stąd wykorzystując \eqref{entropia_izolowany}:
	\begin{equation}
		dS_{uk}=d_{i}S_{uk}
	\end{equation}
	gdzie
	\begin{description}
		\item[\(dS_{uk}\)] całkowita zmiana entropii układu
		\item[\(d_{e}S=\frac{\dbar Q}{T_{uk}}\)] zmiana wynikająca z przepływu ciepła
		\item[\(d_{i}S_{uk}\geq 0\)] zmiana wynikająca z produkcji entropii
	\end{description}
	Dla każdego układu produkcja entropii jest dodatnia. Wynika z tego argumentu także, że jest to prawdziłe dla każdego podukładu należącego do danego układu niezależnie od jego wielkości. Zasada ta ma więc charakter lokalny. 
	\subsection{Termodynamika liniowa i nieliniowa}
	W ogólności natężenie przepływów termodynamicznych jest dowolną funkcją bodźców termodynamicznych: 
	\[J=f\left(X\right)\]
	Rozwinięcie w szereg Taylora tej funkcji wokół \(X_{0}\) jest
	\begin{equation}\label{ogolne_natezenie}
		J_{i}=J_{i}^{eq}+\sum_{k=1}^{n}\left[\frac{\partial J_{i}}{\partial X_{k}}\left(X_{k}-X_{k}^{eq}\right)\right]+\sum_{k=1}^{n}\left[\frac{1}{2}\frac{\partial^{2} J_{i}}{\partial X_{k}^{2}}\left(X_{k}-X_{k}^{eq}\right)^{2}\right]+\ldots
	\end{equation}
	gdzie:
	\begin{description}
		\item[\(J\), \(J^{eq}\)] natężenie przepływów termodynamicznych oraz to natężenie w stanie równowagi
		\item[\(X\), \(X^{eq}\)] bodziec termodynamiczny i bodziec w stanie równowagi
	\end{description}
	Wiadomo, że \(J_{i}^{eq}=0\) oraz \(X_{k}^{eq}=0\), ponieważ jest to stan równowagi. \par
	W stanach zbliżonych do stanu równowagi można ograniczyć równanie \eqref{ogolne_natezenie} do następującego: 
	\begin{equation}\label{natezenie_liniowe}
		J_{i}=\sum_{k=1}^{n}\left[\frac{\partial J_{i}}{\partial X_{k}}X_{k}\right]
	\end{equation}
	Zapiszmy \(\frac{\partial J_{i}}{\partial X_{k}}\) jako \(L_{ik}\)
	\begin{equation}
		J_{i}=\sum_{k=1}^{n}L_{ik}X_{k}
	\end{equation}
	Natężenie przepływu może zależeć tylko od bodźca sprzężonego jak w prawie Fouriera \(\left(\boldsymbol{J}_{q}=-k\nabla T\right)\), Ficka \(\left(\boldsymbol{J}_{C}=-D\nabla C\right)\), są to wtedy procesy proste \cite{orlik}. Mogą one też zależeć od innych bodźców, przykładowo efekt Seebecka oraz Peltiera:
	\begin{gather}
		\boldsymbol{Q}=L_{qq}\Delta T+L_{qI}\Delta \phi \\
		\boldsymbol{I}=L_{Iq}\Delta T+L_{II}\Delta \phi
	\end{gather}\cite{Ceynowa2008}
	Występują w nich procesy krzyżowe; różnica temperatury wywołuje przepływ prądu oraz różnica potencjału elektrycznego wywołuje przepływ ciepła.
	Okazuje się, że współczynniki krzyżowe są sobie równe: \(L_{qI}=L_{Iq}\). Jest to reguła przemienności Onsagera, która została udowodniona doświadczalnie oraz na podstawie fizyki statystycznej. 
	% produkcja entropii
	Jednak w chemii termodynamika liniowa nie opisuje dobrze szybkości zachodzenia reakcji chemicznych, których prędkość zazwyczaj zależy od wyższych potęg stężenia związków. \par
	
	Termodynamika liniowa ma zastosowanie w większości przypadków technicznych, jednak nie są one w stanie opisać reakcji chemicznych, których szybkość zależy od wyższych niż 1 potęg stężenia składników. \par
	Po zaprzestaniu wymuszania nierównowagi układ dąży do stanu równowagi z minimalną zerową produkcją entropii. Przy utrzymaniu stałego bodźca termodynamicznego układ dąży do stanu stacjonarnego o minimalnej dodatniej produkcji entropii. \par
	Rozwijając entropię \(s\left(E\right)\) po zakłóceniu \(\delta E\)w szereg Taylora otrzymujemy:
	\begin{equation}
		S=S^\circ+\left(\frac{\partial S}{\partial E}\right)^\circ \delta E+\frac{1}{2}\left(\frac{\partial^{2}S}{\partial E^{2}}\right)^\circ \left(\delta E\right)^2 + \ldots
	\end{equation}
	W stanie równowagi \(\left(\frac{\partial S}{\partial E}\right)^\circ=0\). Ograniczając szereg do wyrazu drugiego rzędu mamy:
	\begin{equation}
		S=S^\circ+\frac{1}{2}\left(\frac{\partial^{2}S}{\partial E^{2}}\right)^\circ \left(\delta E\right)^2=S^\circ+\frac{1}{2} \delta E\left(\frac{\partial \delta S}{\partial E}\right)^\circ=S^\circ+\frac{1}{2}\delta^{2}S
	\end{equation}
	Ze względu na maksimum entropii w stanie równowagi: \(\left(\delta^2 S\right)^\circ<0\)
	Można powiązać tą zależność z generacją entropii: 
	\begin{equation}
		\frac{\partial}{\partial t}\left(S-S^\circ\right)=\frac{1}{2}\frac{\partial}{\partial t}\delta^2 S=\sum_{i=1}^{n}J_{i}X_{i}=\sigma\geq0
	\end{equation}
	
	
	\subsubsection{Podpodsekcja 1.1.1}
	%\blindtext
	\section{Sekcja 2}
	%\blindtext
	\section{Sekcja 3}
	%\blindtext
	\section{Sekcja 4}
	%\blindtext 
	\pagebreak
	%\section*{Wykaz literatury}
	\addcontentsline{toc}{section}{Wykaz literatury}
	\printbibliography[title=Wykaz literatury]
	\pagebreak
	\section*{Wykaz rysunków}
	\addcontentsline{toc}{section}{Wykaz rysunków}
	\pagebreak
	\section*{Wykaz tabel}
	\addcontentsline{toc}{section}{Wykaz tabel}
	\pagebreak
	\section*{Dodatek A}
	\addcontentsline{toc}{section}{Dodatek A}
	\pagebreak
\end{document}