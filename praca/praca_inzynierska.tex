\documentclass[10pt, a4paper, twoside, onecolumn]{article}
\usepackage[backend=biber, sorting=none, maxbibnames=999]{biblatex}
\addbibresource{refs.bib}
\usepackage[utf8]{inputenc}
\usepackage[T1]{fontenc}
\usepackage[polish]{babel}
\usepackage{uarial}
\usepackage[top=2.5cm,bottom=2.5cm,inner=3.5cm,outer=2.5cm]{geometry}
\usepackage{fancyhdr}
\usepackage{indentfirst}
\usepackage{graphicx}
\usepackage{hyperref}
\usepackage{float}
\usepackage{multirow}
\usepackage{amsmath}
\usepackage{amssymb}
\usepackage{amsthm}
\usepackage{sectsty}
\usepackage{etoolbox}
\usepackage[font=small]{caption}
\usepackage{subcaption}
\usepackage{physics}
\usepackage[version=4]{mhchem}
\usepackage{gensymb}
\usepackage{mathrsfs}
\usepackage{xfrac}
\usepackage{array}
\usepackage{esint}
\usepackage{tabularx}

%\usepackage{titling}
\usepackage{anyfontsize}
\usepackage{blindtext}

\graphicspath{ {./} }
\setlength{\parindent}{1.25cm}
\setlength{\parskip}{12pt}

\pagestyle{fancy}
\fancyhf{}
\fancyfoot[C]{\fontfamily{ua1}\fontsize{9pt}{9pt}\selectfont\thepage}
\renewcommand{\headrulewidth}{0pt}
\renewcommand{\footrulewidth}{0pt}

\renewcommand{\familydefault}{ua1}
\renewcommand{\baselinestretch}{1.5}

\setcounter{page}{3}
\raggedbottom

\sectionfont{\noindent\fontsize{12}{15}\selectfont}
\subsectionfont{\noindent\fontsize{10}{15}\selectfont\textit}
\subsubsectionfont{\noindent\fontsize{10}{15}\selectfont\normalfont\textit}

\BeforeBeginEnvironment{tabular}{\small}

\numberwithin{equation}{section}

\newcommand{\dbar}{d\hspace*{-0.08em}\bar{}\hspace*{0.1em}}

\newtheorem*{theorem}{Twierdzenie}
\newtheorem*{definition}{Definicja}
\addto\captionspolish{\renewcommand{\listfigurename}{Wykaz rysunków}}
\addto\captionspolish{\renewcommand{\listtablename}{Wykaz tabel}}

%\newcolumntype{L}{>{\(}l<{\)}}
%\newcolumntype{C}{>{\(}c<{\)}}
%\newcolumntype{R}{>{\(}r<{\)}}
\newcolumntype{Y}{>{\small\centering\arraybackslash}X}

% Problemy:
% W spisie kropki przy section
% Odpowiednie przerwy miedzy akapitami
% Opisy tabel nad tabela
% Odpowiednie przerwy miedzy opisem a tabela/rysunkiem
% Byc moze problemy z numerowaniem tabel, ale chyba jest na razie dobrze

\begin{document}
	\section*{Streszczenie}
	%\begin{center}
	%	\textbf{Strona 3}
	%\end{center}
	%Termodynamika równowagowe wymaga założenia pewnych wyidealizowanych założeń, 
	%Termodynamika równowagowa oparta jest na kilku wyidealizowanych założeniach. Są nimi na przykład jednorodność parametrów intensywnych, maksimum entropii w układach izolowanych, minimum energii swobodnej Helmholtza w układach izotermiczno-izochorycznych, mininum entalpii swobodnej Gibbsa w układach izotermiczno-izobarycznych \cite{pudlik}. \par
	Termodynamika równowagowa opisuje procesy, którym podlega dany układ, tak, jakby zachodziły w sposób kwazistatyczny. To oznacza, że proces jest postrzegany jako ciąg stanów równowagi. W takim opisie nie sposób zdefiniować prędkości procesu, gdyż sam jego czas przebiegu zmierza do nieskończoności. Takie założenia nie zawsze są możliwe, przykładowo dla szybko zachodzących zmian jak w przypadku eksplozji, albo dla układów, w których interesuje nas nie tylko stan układu po osiągnięciu przez niego stanu równowagi, ale też jego zachowania podczas tej przemiany. W pracy zająłem się głównie reakcjami chemicznymi oscylacyjnymi, które są przykładem czasowej struktury dyssypatywnej. Struktur takich nie można rozpatrywać w ramach termodynamiki równowagowej, ponieważ są one typowe jedynie dla stanów nierównowagowych. \par
	Celem pracy jest teoretyczna oraz numeryczna analiza oscylacyjnych reakcji chemicznych. W pracy zasymulowałem reakcje oscylacyjne, których mechanizmy opisane są odpowiednio modelami Lotki, Lotki-Volterry oraz bruskelatora. Przeprowadziłem teoretyczną analizę stabilności tych modeli w stanach stacjonarnych oraz zastosowałem różne metody numeryczne, które pozwoliły mi zilustrować trajektorie generowane tymi modelami. Również numerycznie oszacowano produkcję entropii. \par\noindent
	Słowa kluczowe: reakcja oscylacyjna, entropia, struktura dyssypatywna
	%Przeprowadzono teoretyczną analizę stabilności w stanach stacjonarnych oraz sprawdzono jej prawdziwość przy użyciu symulacji. Numerycznie oszacowano produkcję entropii. 
	
	%Celem pracy jest pokazanie, że reakcje oscylacyjne są możliwe oraz że występuje w nich produkcja entropii.
	%W pracy zasymulowałem reakcje oscylacyjne, których mechanizm opisany jest modelami Lotki, Lotki-Volterry oraz bruskelator. Przeprowadzono teoretyczną analizę stabilności w stanach stacjonarnych oraz sprawdzono jej prawdziwość przy użyciu symulacji. 
	
	%Pierwsze takie reakcje odkryto w pierwszej połowie XX w., jednak jednym z nich jest organizm żywy, który można rozpatrywać jako bardzo skomplikowany układ termodynamiczny, w którym zachodzą cykliczne zmiany, na przykład stężenie hormonów oraz bicie serca. Najważniejszą reakcją dla rozwoju teorii reakcji oscylacyjnych jest reakcja Biełousowa-Żabotyńskiego. \par
	\pagebreak
	
	\section*{Abstract}
	%\begin{center}
	%	\textbf{Strona 4}
	%\end{center}
	Equilibrium thermodynamics describes the processes in a system as if they were quasistatic. It means that at every point of the process, the system is in equilibrium. Rate of change of any value is impossible to define because time of equilibrium process approaches infinity. Such assumptions cannot always be met in real cases. For example rapidly changing systems like explosions or for cases in which we do not care only about the state of the system in equilibrium after some process, but also its state during said process. In this thesis I mainly focused on oscillating chemical reactions, which are an example of temporal dissipative structures. Those structures cannot be described using equilibrium thermodynamics because they are typical for non-equilibrium systems. \par
	The objective of my thesis is theoretical and numerical analysis of oscillating chemical reactions. In this thesis I simulated chemical reactions using Lotka, Lotka-Volterra and brusselator models. I theoretically analysed those systems near their equilibrium states and confirmed those results using simulation. I numerically estimated entropy production. \par\noindent
	Keywords: chemical oscillator, entropy, dissipative structure
	\pagebreak
	
	%\section*{Spis treści}
	%\begin{center}
	%	\textbf{Strona 5}
	%\end{center}
	%\addcontentsline{toc}{section}{Spis treści}
	\tableofcontents
	\pagebreak
	
	\section*{Wykaz oznaczeń}
	\addcontentsline{toc}{section}{Wykaz oznaczeń}
	\setlength{\parindent}{0cm}
	\begin{table}[H]
	\begin{tabular}{@{} ll}
		\(T\) & Temperatura \\
		\(S\) & Entropia \\
		\(p\) & Ciśnienie \\
		\(V\) & Objętość \\
		\(A\) & Powinowactwo chemiczne \\
		\(\xi\) & Liczba postępu reakcji \\
		%\(s\) & Entropia na jednostkę objętości \\
		%\(\mathscr{S}\) & Entropia na mol \\
		\(U\) & Energia wewnętrzna \\
		%\(H=U+pV\) & Entalpia \\
		%\(F=U-TS\) & Energia swobodna \\
		%\(G=H-TS\) & Entalpia swobodna \\
		%\(u, h, f, g\) & Energia, Entalpia, Energia swobodna oraz Entalpia swobodna na jednostkę objętości \\
		%\(\mathscr{U, H, F, G}\) & Energia, Entalpia, Energia swobodna oraz Entalpia swobodna na mol \\
		%\(C_{p}, C_{V}\) & Pojemności cieplne (\(c_{p}, c_{V}\) ciepła własciwe) \\
		%\(N\) & Liczba cząsteczek (cząstek) \\
		\(\mu\) & Potencjał chemiczny \\
		\(n_{i}\) & Liczba moli $i$-tego składnika \\
		\(\nu_{i}\) & Współczynnik stechiometryczny $i$-tego składnika \\
		\(\sigma\) & Źródło produkcji entropii
	\end{tabular}
	\end{table}
	\setlength{\parindent}{1.25cm}
	\pagebreak
	
	\section{Wstęp}
	
	Oscylacyjne reakcje chemiczne są przykładem procesu samoorganizacji w układach z reakcją chemiczną. W trakcie przebiegu takiej reakcji możemy zaobserwować oscylacyjne zmiany stężenia niektórych reagentów pojawiających się w czasie jej przebiegu. Zwykle są to przejściowe związki chemiczne, które pojawiają się w mechanizmie reakcji pomiędzy substratami a produktami. Zjawisko takiej samoorganizacji obserwujemy tylko wówczas, gdy układ z reakcją chemiczną jest w stanie dalekim od stanu równowagi termodynamicznej. \par
	Te oscylacyjne zmiany stężenia niektórych reagentów w oscylacyjnej reakcji chemicznej mogą odbywać się jednocześnie i tak samo w całej objętości układu, wówczas mówimy o powstaniu czasowej struktury dyssypatywnej. Jeśli stężenia tych reagentów zmieniają się zarówno w czasie jak i przestrzeni, wówczas mówimy o czasowo-przestrzennych strukturach dyssypatywnych. W tym drugim przypadku zaobserwujemy falę stężenia reagenta, która będzie przemieszczać się poprzez całą objętość układu. W trzecim przypadku zmiany stężeń dotyczą tylko objętości układu, są niezależne od czasu, wówczas mówimy o pojawieniu się przestrzennej struktury dyssypatywnej. Wraz z osiągnięciem równowagi termodynamicznej w układzie opisane struktury zanikają. 
	
	% Wkleić przypadki struktur
	\begin{figure}[H]
		\centering
		\begin{subfigure}{0.45\textwidth}
			\centering
			\includegraphics[width=\textwidth]{"belousov_zhabotinsky".png}
			\caption{Fale chemiczne w reakcji Biełousowa-Żabotyńskiego (po lewej) oraz śluzowiec Dictyostelium discoideum (po prawej) \cite{epstein}}
			\label{fig:bz_disco}
		\end{subfigure}
		\begin{subfigure}{0.45\textwidth}
			\centering
			\includegraphics[width=\textwidth]{"benard".png}
			\caption{Komórki B\'{e}narda \cite{prigogine}}
			\label{fig:benard}
		\end{subfigure}
		\caption{Przykładowe struktury dyssypatywne}
		\label{fig:struktury_dyssypatywne}
	\end{figure}
	
	Na rysunkach \ref{fig:bz_disco} oraz \ref{fig:benard} przedstawiono przykładowe struktury dyssypatywne. Na rysunku \ref{fig:bz_disco} po lewej stronie możemy zobaczyć typowe fale chemiczne powstałe w trakcie przebiegu reakcji Biełousova-Żabotyńskiego. Natomiast po prawej stronie rysunku \ref{fig:bz_disco} mamy analogiczną strukturę, lecz stworzoną przez jednokomórkowe śluzowce Dictyostelium discoideum, które są w trakcie formowania swojej wielokomórkowej postaci - superorganizmu. Pokazuje to, że podobne wizualnie struktury występują zarówno w chemii jak i biologii. Na rysunku \ref{fig:benard} przedstawiono komórki B\'{e}narda, które występują w zjawisku konwekcji cieczy lepkiej po osiągnięciu różnicy temperatur między dwoma płytkami większej od pewnej temperatury krytycznej. Płytki są prostopadłe do siły ciążenia, a cieplejsza z nich jest pod tą zimniejszą. \par
	Wyjaśnienie pojawienia się struktur dyssypatywnych z punktu widzenia termodynamiki wymaga wyjścia poza termodynamikę równowagową. Termodynamika równowagowa zajmuje się procesami, w których ignoruje się upływ czasu, a przemiana jest kwazistatyczna. Oznacza to, że każdy stan pośredni można traktować jako stan równowagi termodynamicznej. Model taki jest wystarczający do opisu większości procesów, %Można więc powiedzieć, że termodynamikę równowagową 
	w których interesuje nas stan początkowy oraz końcowy. Jest jednak niewystarczający, jeżeli interesuje nas szybkość zachodzenia procesu. Dopiero termodynamika nierównowagowa jest teorią, która obejmuje to, co dzieje się w trakcie rzeczywistych przemian i jest ona konieczna do opisu reakcji oscylacyjnych. Pierwsze odkrycia tego typu przemian sięgają końca XIX wieku. Były to reakcje w układach heterogenicznych, jak na przykład pierścienie Lieseganga lub oscylacje prądu płynącego przez ogniwo galwaniczne \cite{orlik}. \par
	% Wyjaśnienie tych zjawisk wymagało, aby układ byl heterogeniczny i było w zgodzie z entropią Boltzmanna, według której spontaniczna organizacja jest niemożliwa. \cite{orlik}
	Pierwszy model teoretyczny reakcji oscylacyjnej został przedstawiony przez Alfreda Lotkę \cite{lotka}. W modelu tym zakłada się, że reakcje przebiegają w układzie homogenicznym. Przez długi czas uważano, że nie może on przedstawiać rzeczywistych reakcji, ponieważ ówcześnie interpretowano, że łamie drugą zasadę termodynamiki \cite{orlik_sily_w_przyrodzie}. Jednak w 1921r. na przykładzie reakcji Bray'a-Liebhafky'ego pokazano, że reakcje oscylacyjne w układach homogenicznych są możliwe. Jest to reakcja rozkładu nadtlenku wodoru katalizowana jodanem (V). W trakcie jej przebiegu obserwujemy naprzemienną dominację dwóch reakcji \cite{orlik}:
	\begin{center}
		\ce{5H2O2 + 2H+ +2IO3- -> I2 + 5O2 + 6H2O}
	\end{center}
	oraz
	\begin{center}
		\ce{I2 + 5H2O2 -> 2H+ + 2IO3- + 4H2O}.
	\end{center}
	Oznacza to między innymi, że będziemy obserwować oscylacyjne zmiany stężenia \ce{I2} oraz szybkości wydzielania \ce{O2}, gdyż w pierwszej reakcji następuje produkcja \ce{I2}, a w drugiej następuje jego zużycie. Reakcja sumaryczna w tej przemianie ma postać: 
	\begin{center}
		\ce{2H2O2 -> 2H2O + O2}.
	\end{center}
	
	Jeszcze większy wpływ na rozwój termodynamiki nierównowagowej w opisie przemian chemicznych miało odkrycie reakcji Biełousowa-Żabotyńskiego. Pierwszą reakcją z tej grupy została zaobserwowanaw 1959 w wodnym roztworze bromianu (V) potasu, siarczanu (VI) ceru (IV), kwasu malonowego lub kwasu cytrynowego z dodatkiem rozcieńczonego kwasu siarkowego (VI). Reakcje te przebiegają według dość skomplikowanego mechanizmu, wciąż badanego, ale można podać uproszczone równania tej przemiany \cite{orlik}: %Reakcje te są dużo bardziej złożone i w uproszczonej formie mają postać 
	\begin{center}
		\ce{2Br- + BrO3- + 3CH2(COOH)2 + 3H+ -> 3BrCH(COOH)2 + 3H2O}, \\
		\ce{BrO3- + 4Ce^3+ + 5H+ -> 4Ce^4+ + HOBr + 2H2O}, \\
		\ce{4Ce^4+ + 6CH2(COOH)2 +BrCH(COOH)2 + HOBr + 8BrO3- -> 4Ce^3+ + 10Br- + 21CO2 + 6H+ + 11H2O}.
	\end{center}
	W reakcji tej obserwujemy, że stężenia \ce{Br-} oraz \ce{HBrO2}, z których drugi nie jest obecny w przedstawionych wyżej sumarycznych równaniach, zmieniają się oscylacyjnie. Równocześnie stężenia \ce{Ce^{3+}} oraz \ce{Ce^{4+}} też podlegają oscylacjom. Oscylacje stężenia tego reagenta powoduje widoczne zmiany koloru roztworu w trakcie reakcji z bezbarwnego na żółty, a później ponownie na bezbarwny. Całkowita reakcja sumaryczna ma postać: 
	\begin{center}
		\ce{2BrO3- + 3CH2(COOH)2 + 2H+ ->[kat] 2BrCH(COOH)2 + 4H2O + 3CO2}.
	\end{center}
	% Została ona odkryta jako nieorganiczny analog cyklu Krebsa \cite{belousov_hist}. 
	W późniejszym okresie znaleziono również inne reagenty, dla których zachodzi reakcja analogiczna. Jedną z możliwych modyfikacji tej reakcji jest zamiana jonów \ce{Ce^3+} na \ce{Fe(o-phen)3^2+}, zwanych ferroiną, której wzór strukturalny jest umieszczony na rysunku \ref{fig:ferroine}. Jej utleniona forma ma wzór \ce{Fe(o-phen)3^3+}. Podmiana ta zmienia efekty wizualne w czasie przebiegu reakcji oscylacyjnej. Zamiast oscylacji między bezbarwnym i żółtym roztworem występuje oscylacja między pomarańczowym oraz niebieskim \cite{orlik}. Zamiast ferroiny można użyć również związku rutenu \ce{Ru(bpy)3^2+}, którego formę utlenioną zapiszemy wzorem \ce{Ru(bpy)3^3+}. Wzór strukturalny \ce{Ru(bpy)3^2+} przedstawiono na rysunku \ref{fig:rubpy}. Oscylacje stosunku stężeń tych katalizatorów mogą się też objawiać w inny sposób niż zmiana koloru. \cite{osypova}\par
	\begin{figure}[H]
		\centering
		\begin{subfigure}{0.45\textwidth}
			\centering
			\includegraphics[width=\textwidth]{"ferroine".png}
			\caption{\ce{Fe(o-phen)3^2+}\cite{sigmaaldrich}}
			\label{fig:ferroine}
		\end{subfigure}
		\begin{subfigure}{0.45\textwidth}
			\centering
			\includegraphics[width=\textwidth]{"ru(bpy)3".png}
			\caption{\ce{Ru(bpy)3^2+}\cite{sigmaaldrich}}
			\label{fig:rubpy}
		\end{subfigure}
		\caption{Przykładowe katalizatory w reakcji Biełousowa-Żabotyńskiego}
	\end{figure}
	Odkrycie reakcji oscylacyjnych przyczyniło się do rozwoju termodynamiki nierównowagowej, a w konsekwencji do użycia jej aparatu teoretycznego do opisu procesów życiowych. 
	Reakcja Biełousowa-Żabotyńskiego traktowana jest czasem jako analog cyklu Krebsa, będącego głównym źródłem energii organizmów tlenowych. Analogia ta dotyczy mechanizmu reakcji, w których występuje pętla sprzężenia zwrotnego, to oznacza, że produkt pierwszej reakcji jest substratem kolejnej, aż do osiagniecia produktu, który jest substratem tej pierwszej \cite{belousov_hist, stryer}. W cyklu Krebsa mogą również wystąpić oscylacje tak jak w reakcji Biełousowa-Żabotyńskiego \cite{krebs_oscillations}. \par
	Opisane wyżej struktury dyssypatywne są przykładem samoorganizacji materii, która może się pojawić w warunkach dalekich od stanu równowagi termodynamicznej. Niepoprawna interpretacja drugiej zasady termodynamiki może doprowadzić do wniosku, że powstanie opisanych wyżej złożonych struktur z molekularnego chaosu powinno być niemożliwe. Tak rzeczywiście się dzieje, ale tylko dla układów izolowanych. Natomiast w przypadku innych układów, które oddziałują z otoczeniem samoorganizacja materii jest możliwa. Przykładem są procesy zachodzące w układach biologicznych, w którcyh interakcja z otoczeniem jest kluczowa. %Wyjaśnienie takie błędnie wykorzystuje tę zasadę ignorując fakt, że układy biologiczne jak i cała Ziemia nie są układami izolowanymi. 
	Wspomniana wyżej błędna interpretacja drugiej zasady termodynamiki polega na ekstrapolowaniu wyników z układów izolowanych i przemian równowagowych do wszystkich innych przemian.
	
	%Opisane wyżej struktury dyssypatywne są przykładem samoorganizacji materii, która może się pojawić w warunkach dalekich od stanu równowagi termodynamicznej. Samoorganizująca się materia oddziałując z otoczeniem musi w. Przykładami są tutaj układy biologiczne, dla których
	
	\subsection{Produkcja entropii w układach nierównowagowych}
	Wszystkie procesy rzeczywiste podlegają drugiej zasadzie termodynamiki, która określa różniczkową zmianę entropii \(\dd{S}\), w danych procesie samorzutnym nierównością \cite{buchowski}:
	\begin{equation}\label{II_Zasada_Termodynamiki}
		\dd{S}>0.
	\end{equation}
	W powyższym zapisie pod symbolem \(\dd{S}\) kryje się suma różniczkowych zmian entropii otoczenia \(\dd{S_{ot}}\) i układu \(\dd{S_{uk}}\)
	\begin{equation}\label{entropia_uklad_otoczenie}
		\dd{S}=\dd{S_{uk}}+\dd{S_{ot}}.
	\end{equation}
	W przypadku układów izolowanych \(\dd{S}=\dd{S_{uk}}\), 
	więc w układach izolowanych entropia zawsze rośnie, a zatem spontaniczne uporządkowanie stabilnych struktur nie jest możliwe. \par
	%Druga zasada termodynamiki dana wzorem \eqref{II_Zasada_Termodynamiki} rozstrzyga czy dany proces jest możliwy. 
	Aby nadać tej zasadzie ilościowy charakter, w której zastąpimy nierówność równością, wprowadza się pojęcie produkcji entropii, które przeanalizujemy najpierw z punktu widzenia układu. W przypadku układu zamkniętego, w którym przebiega proces samorzutny drugą zasadę termodynamiki zapiszemy następującą nierównością \cite{orlik, pigon1, jaworski}: 
	\begin{equation}\label{II_Zasada_Termodynamiki_uklad}
		\dd{S_{uk}}>\frac{\dbar Q}{T_{ot}},
	\end{equation}
	gdzie \(\dd{S_{uk}}\) to różniczkowa zmiana entropii układu, \(\dbar Q\) to elementarne ciepło dostarczone do układu ze źródła o temperaturze \(T_{ot}\). Jeśli wymiana \(\dbar Q\) odbywa się w temperaturze \(T\) to 
	\begin{equation}\label{II_Zasada_Termodynamiki_uklad_T}
		\dd{S_{uk}}>\frac{\dbar Q}{T}.
	\end{equation}
	Nierówność \eqref{II_Zasada_Termodynamiki_uklad} jest konsekwencją podstawienia \(\dd{S_{ot}}=-\frac{\dbar Q}{T_{ot}}\) oraz \eqref{entropia_uklad_otoczenie} do \eqref{II_Zasada_Termodynamiki}, natomiast \eqref{II_Zasada_Termodynamiki_uklad_T} otrzymujemy podstawiając \(\dd{S_{ot}}=-\frac{\dbar Q}{T}\) do tych samych równań. \par
	%Nierówności \eqref{II_Zasada_Termodynamiki_uklad} i \eqref{II_Zasada_Termodynamiki_uklad_T} są konsekwencją podstawienia \(\dd{S_{ot}}=-\frac{\dbar Q}{T_{ot}}\) lub \(\dd{S_{ot}}=-\frac{\dbar Q}{T}\) do \eqref{II_Zasada_Termodynamiki} oraz \eqref{entropia_uklad_otoczenie}. 
	% Z równania \eqref{II_Zasada_Termodynamiki} lub \eqref{II_Zasada_Termodynamiki_uklad} wynika, że w układach izolowanych entropia zawsze rośnie, a zatem spontaniczne uporządkowanie stabilnych struktur nie jest możliwe.
	Z nierówności \eqref{II_Zasada_Termodynamiki_uklad} lub \eqref{II_Zasada_Termodynamiki_uklad_T} wynika, że różniczkową zmianę entropii układu \(\dd{S_{uk}}\) możemy przedstawić jako sumę dwóch wkładów:
	\begin{equation}
		\dd{S_{uk}}=\dd_{e}S+\dd_{i}S,
	\end{equation}
	gdzie \(\dd_{e}S=\frac{\dbar Q}{T}\), stanowi wkład do \(\dd{S_{uk}}\) wynikający tylko z wymiany ciepła \(\dbar Q\), \(\dd_{i}S\) jest równy:
	\begin{equation}
		\dd_{i}S=\dd{S_{uk}}-\frac{\dbar Q}{T}.
	\end{equation}
	Składnik \(\dd_{i}S\) nazywany jest produkcją entropii. Na podstawie nierówności \eqref{II_Zasada_Termodynamiki} lub \eqref{II_Zasada_Termodynamiki_uklad} wynika, że \(\dd_{i}S>0\) w przemianach samorzutnych. \par
	Rozpatrzmy teraz drugą zasadę termodynamiki z punktu widzenia układu i otoczenia, która ze wzorów \eqref{II_Zasada_Termodynamiki} oraz \eqref{entropia_uklad_otoczenie} przyjmuje postać:
	\begin{equation}\label{II_Zasada_Termodynamiki_uklad_i_otoczenie}
		\dd{S_{uk}}+\dd{S_{ot}}>0,
	\end{equation}
	gdzie symbolem \(\dd{S_{ot}}\) oznaczono różniczkową zmianę entropii otoczenia. Zmiany entropii \(\dd{S_{ot}}\) i \(\dd{S_{uk}}\) mogą wynikać z wymiany ciepła \(\dbar Q\) jak i produkcji entropii. Dlatego możemy zapisać:
	% Do nierówności \eqref{II_Zasada_Termodynamiki_uklad_i_otoczenie} wykorzystamy pojęcia zmieny entropii wynikającej z wymiany ciepła pomiędzy układem i otoczeniem oraz produkcji entropii, która może pojawić się w układzie i otoczeniu. Z nierówności \eqref{II_Zasada_Termodynamiki_uklad_i_otoczenie} otrzymujemy:
	\begin{equation}\label{II_Zasada_Termodynamiki_uklad_i_otoczenie_rozpisane}
		\dd_{e}S_{uk}+\dd_{i}S_{uk}+\dd_{e}S_{ot}+\dd_{i}S_{ot}>0.
	\end{equation}
	Jeśli wymiana ciepła zachodzi w temperaturze \(T\) to wówczas: \(\dd_{e}S_{uk}=-\dd_{e}S_{ot}\), otrzymujemy z równania \eqref{II_Zasada_Termodynamiki_uklad_i_otoczenie_rozpisane} 
	\begin{equation}\label{II_Zasada_Termodynamiki_tworzenie}
		\dd_{i}S_{uk}+\dd_{i}S_{ot}>0.
	\end{equation}
	Warunek \eqref{II_Zasada_Termodynamiki_tworzenie} obejmuje również przypadek, w którym produkcja entropii \(\dd_{i}S_{uk}\) zmaleje na tyle, że \(\dd_{i}S_{ot}\) skompensuje ten niedostatek. W innym przypadku \(\dd_{i}S_{ot}\) może na tyle zmaleć, że \(\dd_{i}S_{uk}\) będzie kompensować ten niedostatek. W termodynamice nierównowagowej dokonujemy jednak dodatkowo założenia:
	\begin{equation*}
		\dd_{i}S_{uk}>0
	\end{equation*}
	oraz
	\begin{equation*}
		\dd_{i}S_{ot}>0.
	\end{equation*}
	
	Założenie to możemy przenieść na sytuację, gdy układ dzielimy na mniejsze podukłady (komórki). Z punkty widzenia pojedynczej komórki, dla której pozostałe stanowią otoczenie, oznacza to, że produkcja entropii w jej wnętrzu ma być nieujemna i podobnie dla pozostałych. To oznacza dalej, że w każdym dowolnie małym obszarze układu, w którym zachodzą procesy samorzutne następuje związane z nimi tworzenie entropii. To stwierdzenie stanowi treść hipotezy termodynamiki nierównowagowej, nazywaną lokalnym sformuowaniem drugiej zasady termodynamiki. %Lokalne sformuowanie drugiej zasady termodynamiki 
	Nie wyklucza ona jednak takiego przypadku, w którym w jednym i tym samym miejscu zachodzi kilka procesów, z których niektóre zmniejszają entropię \((\dd_{i}S<0)\), pod warunkiem, że oprócz nich obecne są procesy produkujące entropię \((\dd_{i}S>0)\), które z naddatkiem zwiększa entropię w tym miejscu \cite{orlik, pigon1, guminski}. \par
	Tego typu sprzężenie procesów jest obserwowane w układach nierównowagowych. Przykładem jest tutaj termodyfuzja w gazie. W procesie tym pod wpływem gradientu temperatury dochodzi do przepływu ciepła, co jest procesem nieodwracalnym, wytwarzającym entropię w każdym miejscu. Jednak równocześnie w początkowo jednorodnym układzie zaobserwujemy transport masy, którego rezultatem będzie pojawienie się niejednorodności stężenia gazu w przestrzeni, którą zajmuje. Z tym drugim procesem związane jest zmniejszenie entropii. Wytwarzanie entropii związane z przepływem ciepła większe niż jej spadek związany z wytworzeniem niejednorodności. 
	% Zmianę entropii układu można zapisać jako wynikającą z przepływu ciepła oraz samorzutnej produkcji entropii: 
	% \begin{equation}
	% 	dS_{uk}=d_{e}S+d_{i}S_{uk}.
	% \end{equation}
	% Dla układu izolowanego: \(d_{e}S=0\):
	% \begin{equation}
	% 	dS_{uk}=d_{i}S_{uk}.
	% \end{equation}
	% gdzie
	% \begin{description}
	% 	\item[\(dS_{uk}\)] całkowita zmiana entropii układu
	% 	\item[\(d_{e}S=\frac{\dbar Q}{T_{uk}}\)] zmiana wynikająca z przepływu ciepła
	% 	\item[\(d_{i}S_{uk}\geq 0\)] zmiana wynikająca z produkcji entropii
	% \end{description}
	% Dla każdego układu produkcja entropii jest dodatnia. Wynika z tego argumentu także, że jest to prawdziwe dla każdego podukładu należącego do danego układu niezależnie od jego wielkości. Zasada ta ma więc charakter lokalny. 
	\subsection{Szybkość reakcji chemicznej}
	Reakcje chemiczne można podzielić na dwie kategorie. Pierwsza z nich to reakcje homogeniczne, czyli takie zachodzące w jednej fazie. Druga to reakcje heterogeniczne, czyli reakcje zachodzące między związkami chemicznymi w różnych fazach i na granicy między nimi. Szybkość reakcji heterogenicznych jest trudniejsza do opisania, ponieważ zależy od szybkości dyfuzji, powierzchni rozdzielenia i innych czynników. %do uzupelnienia
	Szybkość reakcji homogenicznej w stałej temperaturze jest funkcją stężeń reagentów. Można ją wyrazić poprzez szybkość zmiany stężenia molowego pojedynczego reagenta $c_{i}$ dla stałej objętości jako: 
	\begin{equation}\label{szybkosc_reakcji_def}
		v=\frac{1}{\nu_i}\dv{c_i}{t}.
	\end{equation}
	Jest to wielkość niezależna od konkretnego reagenta. Różniczkową zmianę $c_{i}$ opisuje wyrażenie
	\begin{equation}
		%v=\frac{1}{\nu_{i}}\dv{c_{i}}{t} \\
		\dd{c_{i}}=\frac{\dd{n_{i}}}{V}
	\end{equation}
	gdzie \(\nu_{i}\) to współczynnik stechiometryczny (ujemny dla substratów, a dodatni dla produktów). \par
	Można tak zdefiniowaną szybkość reakcji powiązać z różniczkową zmianą liczby postępu reakcji \cite{atkins_por}:
	\begin{equation}
		\dd{\xi_{r}}=\frac{\dd{n_{ir}}}{\nu_{i}},
	\end{equation}
	gdzie \(\xi_{r}\) to liczba postępu reakcji i jest niezależna od wyboru składnika. Wstawiając tą zależność do \eqref{szybkosc_reakcji_def} otrzymujemy: 
	\begin{equation}
		v=\frac{1}{V}\dv{\xi_{r}}{t}.
	\end{equation}
	%Dla reacji chemicznej w postaci ogólnej
	Szybkość reakcji chemicznej dla reakcji zapisanej wzorem:
	\begin{equation*}
		x_{1}X_{1}+x_{2}X_{2}+\ldots \rightarrow y_{1}Y_{1}+y_{2}Y_{2}+\ldots
	\end{equation*}
	jest w ogólności funkcją stężeń wszystkich reagentów biorących udział w reakcji chemicznej: 
	%szybkość ta ma ogólną postać:
	\begin{equation*}
		v=f\left(x_{1}, x_{2}, \ldots, y_{1}, y_{2}, \ldots\right).
	\end{equation*}
	Odpowiednie wzory są wyznaczane empirycznie i znacząca część z nich okazuje się mieć prostszą formę tej zależności:
	\begin{equation}\label{szybkosc reakcji}
		v=k x_{1}^{\alpha_1}x_{2}^{\alpha_2}\ldots y_{1}^{\beta_1}y_{2}^{\beta_2}\ldots.
	\end{equation}
	Ustalone empirycznie zależności pomiędzy szybkością reakcji, a stężeniami reagentów są podstawą do tworzenia modeli zachodzenia reakcji chemicznych, ich kinetycznego opisu. Przykładowe reakcje i ich szybkości reakcji zostały przedstawione w Dodatku A na stronie \pageref{sec: dodatek A}. Według równania \eqref{szybkosc reakcji} szybkość reakcji można zmienić przez zmianę stężenia reagentów, które w niej uczestniczą lub zmianę stałej szybkości reakcji. \par 
	%Katalizatorami nazywamy związki zwiększające stałą reakcji sumarycznej, natomiast \linebreak związki zmniejszające katalizatorem ujemnym lub inhibitorem. Mają one działanie selektywne, a więc zwiększają szybkość tylko dla pewnej z wielu możliwych reakcji chemicznych dla danych substratów. Reakcja katalizowana przebiega więc innym mechanizmem niż niekatalizowana. Mechanizmem reakcji chemicznej nazywamy ciąg reakcji elementarnych wyjaśniający dokładny przebieg tej reakcji. 
	Reakcje chemiczne przebiegają według tak zwanego mechanizmu reakcji chemicznej. Opisuje on wszystkie jej etapy. Uwzględnia on procesy, które przebiegają w trakcie reakcji wraz z pojawieniem się i znikaniem reagenta pośredniego. Reakcja katalizowana przebiega innym mechanizmem niż niekatalizowana, choć substraty i produkty obu są identyczne. Katalizator, otwierając nową drogę przemiany, powoduje to, że szybciej i wydajniej otrzymujemy produkty. Katalizatory nie występują w reakcji sumarycznej, ale występują w reakcjach elementarnych. Podobnie jak reakcje możemy podzielić katalizę na homogeniczną (zachodzącej w jednej fazie) oraz heterogeniczną (zachodząca na granicy faz). Szczególnym rodzajem reakcji katalitycznych jest autokataliza, w której produkt reakcji bierze w niej udział, przez co zwiększa jej szybkość \cite{pigon1, atkins_por}. 
	%Przykładem takiej reakcji jest hydroliza estrów w środowisku kwasowym: 
	%\begin{center}
	%	\ce{RCOOC2H5 + 2H2O ->[H3O+] RCOO- + C2H5OH + H3O+}
	%\end{center}
	
	\subsection{Termodynamika nierównowagowa liniowa i nieliniowa}
	%Wprowadza się wielkość zwaną źródłem entropii \(\sigma\), która oznacza produkcję entropii w jednostce czasu i objętości
	%\begin{equation}
	%	\pdv{_{i}S}{t} = \iiint_{V}\sigma\dd{V}.
	%\end{equation}
	
	We wcześniejszym podrozdziale opisaliśmy zmiany entropii układu, które dokonują się w trakcie procesu samorzutnego. Jeżeli zmiany te odniesiemy do bardzo krótkiego przedziału czasu \(\dd{t}\), wówczas szybkość zmiany entropii dana będzie za pomocą równania \cite{mazur, prigogine}:
	\begin{equation}
		\dv{S}{t} = \dv{_{e}S}{t} + \dv{_{i}S}{t},
	\end{equation}
	gdzie pierwszy człon po prawej stronie równości opisuje szybkość wymiany entropii układu z otoczeniem, a drugi szybkość produkcji entropii w układzie. Produkcję tę możemy opisać za pomocą pojęcia źródła entropii \(\sigma\). Jest to wielkość produkcji entropii w odniesieniu na jednostkę objętości. W ogólności to funkcja czasu i położenia. Relacja pomiędzy źródłem entropii, a szybkością produkcji entropii określa wzór:
	\begin{equation}
		\dv{_{i}S}{t} = \iiint_{V}\sigma\dd{V},
	\end{equation}
	w którym \(\sigma>0\). \par
	Chcąc przenieść wprowadzone pojęcie na poziom opisu, który pojawia się w lokalnym sformuowaniu drugiej zasady termodynamiki, rozważmy mały fragment układu. Dla tego fragmentu zmiany entropii również odbywają się poprzez wymianę jej przez ścianki ograniczające rozważany fragment oraz z tworzenia jej wewnątrz tego fragmentu. Tę wymianę entropii z otoczeniem opisuje wektor przepływu entropii \(\vb{J}_{s}\). Jest on zależny od położenia fragmentu i czasu. Biorąc pod uwagę pojęcie źródła entropii \(\sigma\) wzór opisujący zmianę w czasie entropii zapiszemy jako:
	\begin{equation} \label{dod A: 5}
		\dv{S}{t} = -\div{\vb{J}_{S}} + \sigma.
	\end{equation}\par
	% Wyprowadzeie wzoru na źródło entropii:
	Oprócz zmieniającej się entropii, w układzie, w ogólności, dochodzi do zmiany energii wewnętrznej. Różniczkową jej zmianę $\dd{U}$ zapiszemy poprzez wyrażenie \cite{prigogine_modern, landau}:
	\begin{equation}
		\dd{U}=T\dd{S}-p\dd{V}+\sum_{i}\mu_{i}\dd{n_{i}}, \label{dod A: 1}
	\end{equation}
	gdzie symbolem $\mu_{i}$ oznaczono potencjał chemiczny $i$-tej substancji. 
	\begin{equation}
		\mu_{i}=\left(\pdv{U}{n_{i}}\right)_{S, V, n_{j}: j\neq i}
	\end{equation}
	Przekształcając równanie \eqref{dod A: 1}, zakładając że $\dd{V}=0$, otrzymujemy \cite{pigon1}:
	\begin{equation}
		\dd{S}=\frac{1}{T}\dd{U}-\sum_{i}\frac{\mu_{i}}{T}\dd{n_{i}}. \label{dod A: 2}
	\end{equation}
	Różniczkowa zmiana \eqref{dod A: 2} na przyrost czasu \(\dd{t}\) ma postać:
	\begin{equation}
		\dv{S}{t}=\frac{1}{T}\dv{U}{t}-\sum_{i}\frac{\mu_{i}}{T}\dv{n_{i}}{t}. \label{dod A: 3}
	\end{equation}
	Jeżeli zmiany te odniesiemy do elementarnej ścianki \(\dd{\vb{a}}\), wówczas otrzymujemy relację pomiędzy gęstością strumienia entropii \(\vb{J}_{S}\), gęstością strumienia energii wewnętrznej \(\vb{J}_{U}\) oraz gęstością strumienia liczby moli \(\vb{J}_{i}\), którą opiszemy równaniem:
	\begin{equation}
		\vb{J}_{S}=\frac{1}{T}\vb{J}_{U}-\sum_{i}\frac{\mu_{i}}{T}\vb{J}_{i}. \label{dod A: 4}
	\end{equation}
	%Równania ciągłości dla entropii:
	%\begin{equation}
	%	\dv{S}{t}=-\div{\vb{J}_{S}}+\sigma, \label{dod A: 5}
	%\end{equation}
	Energia wewnętrzna $U$ jest skalarem, który podlega równaniu ciągłości:
	\begin{equation}
		\dv{U}{t}=-\div{\vb{J}_{U}} \label{dod A: 6}
	\end{equation}
	natomiast szybkość zmiany $n_{i}$ dana jest wzorem:
	\begin{equation}
		\dv{n_{i}}{t}=-\div{\vb{J}_{i}}+\dv{n_{i;reak}}{t}. \label{dod A: 7}
	\end{equation}
	Różniczkowa zmiana liczby moli $i$-tego składnika \(\dd{n_{i;reak}}\) możemy zapisać poprzez liczbę postępu reakcji $\xi_{r}$ oraz współczynnik stechiometryczny $\nu_{i}$. Infinitezymalna zmiana liczby postępu reakcji jest zdefiniowana jako \(\dd{\xi_{r}}=\frac{\dd{n_{ir;reak}}}{\nu_{ir}}\), a roszerzając to do wielu równoległych reakcji przebiegających w roztworze otrzymujemy \(\dd{n_{i;reak}}=\sum\limits_{r}\nu_{ir}\dd{\xi_{r}}\). Zależności tą podstawiamy do równania \eqref{dod A: 7} i otrzymujemy:
	\begin{equation}
		\dv{n_{i}}{t}=-\div{\vb{J}_{i}}+\sum_{r}\nu_{ir}\dv{\xi_{r}}{t}. \label{dod A: 8}
	\end{equation}
	Podstawiając \eqref{dod A: 4}, \eqref{dod A: 6} oraz \eqref{dod A: 8} do \eqref{dod A: 5} otrzymujemy: 
	\begin{equation} \label{dod A: 9}
		\dv{S}{t}=\frac{1}{T}\dv{U}{t}-\sum_{i}\frac{\mu_{i}}{T}\dv{n_{i}}{t}-\left[\vb{J}_{U}\vdot\grad{\frac{1}{T}}-\sum_{i}\vb{J}_{i}\cdot\grad{\frac{\mu_{i}}{T}}-\frac{1}{T}\sum_{r}\sum_{i}\nu_{ir}\mu_{i}\dv{\xi_{r}}{t}\right]+\sigma.
	\end{equation}
	Wykorzystując pojęcie powinowactwa chemicznego: 
	\begin{equation}
		A_{r}=-\sum_{i}\nu_{ir}\mu_{i} \\
	\end{equation}
	równanie \eqref{dod A: 9} przybiera postać:
	\begin{equation}
		\dv{S}{t}=\frac{1}{T}\dv{U}{t}-\sum_{i}\frac{\mu_{i}}{T}\dv{n_{i}}{t}-\left[\vb{J}_{U}\vdot\grad{\left(\frac{1}{T}\right)}-\sum_{i}\vb{J}_{i}\cdot\grad{\left(\frac{\mu_{i}}{T}\right)}+\sum_{r}\frac{A_{r}}{T}\dv{\xi_{r}}{t}\right]+\sigma.
	\end{equation}
	%Otrzymujemy z porównania tego wzoru z \eqref{dod A: 3}:
	Porównując otrzymane wyrażenie z równaniem \eqref{dod A: 3} otrzymujemy:
	\begin{equation} \label{dod A: sigma}
		\sigma=\vb{J}_{U}\vdot\grad{\left(\frac{1}{T}\right)}-\sum_{i}\vb{J}_{i}\cdot\grad{\left(\frac{\mu_{i}}{T}\right)}+\sum_{r}\frac{A_{r}}{T}\dv{\xi_{r}}{t}.
	\end{equation}
	Wyrażenie \eqref{dod A: sigma} przedstawia źródło produkcji entropii $\sigma$ jako sumę iloczynu sił termodynamicznych oraz powodowanych przez nie przepływów. Pominęliśmy w nim jednak produkcję entropii wynikającą z lepkości płynu. Odpowiednie siły oraz związanie z nimi przepływy zestawiono w tabeli \ref{tab: sily i przeplywy}.
	\begin{table}[H]
	\centering
	\begin{tabular}{|l|c|c|}
		\hline
		Proces & Przepływ & Siła termodynamiczna \\
		\hline
		Transport energii & \(\vb{J}_{U}\) & \(\grad{\left(\frac{1}{T}\right)}\) \\
		Dyfuzja & \(\vb{J}_i\) & \(-\grad{\left(\frac{\mu_{i}}{T}\right)}\) \\
		Reakcja chemiczna & \(J_{r}=\dv{\xi_{r}}{t}\) & \(\frac{A_r}{T}\) \\
		\hline
	\end{tabular}
	\caption{Siły i przepływy termodynamiczne}
	\label{tab: sily i przeplywy}
	\end{table}\noindent
	Na podstawie wzoru \eqref{dod A: sigma} źródło produkcji entropii możemy zapisać ogólnym wzorem:
	%Produkcja entropii jest wyrażona za pomocą sił termodynamicznych i przepływów w ogólności jako: 
	\begin{equation}
		\sigma = \sum_{i}J_{i}X_{i} = \sum_{i}\sum_{j}^{n}L_{ij}X_{i}X_{j},
	\end{equation}
	gdzie $X_{k}$ i $J_{k}$ są skalarami, składowymi wektorów lub składowymi tensorów (dla przepływu lepkiego) \cite{guminski_petelenz}. 
	
	W ogólności natężenie przepływów termodynamicznych jest dowolną funkcją sił termodynamicznych: 
	\[J=f\left(X\right)\]
	Rozwinięcie w szereg Taylora tej funkcji wokół \(X^{eq}\) jest
	\begin{equation}\label{ogolne_natezenie}
		J_{i}=J_{i}^{eq}+\sum_{j=1}^{n}\left[\pdv{J_{i}}{X_{j}}\left(X_{j}-X_{j}^{eq}\right)\right]+\frac{1}{2!}\sum_{j=1}^{n}\sum_{k=1}^{n}\left[\pdv{J_{i}}{X_{j}}{X_{k}}\left(X_{j}-X_{j}^{eq}\right)\left(X_{k}-X_{k}^{eq}\right)\right]+\ldots,
	\end{equation}
	gdzie symbolami \(J\), \(J^{eq}\) oznaczono odpowiednio natężenie przepływów termodynamicznych oraz to natężenie w stanie równowagi, natomiast \(X\), \(X^{eq}\) oznaczają odpowiednio bodziec termodynamiczny i bodziec w stanie równowagi.
	
	W stanie równowagi \(J_{i}^{eq}=0\) oraz \(X_{j}^{eq}=0\). W stanach zbliżonych do tego stanu można ograniczyć równanie \eqref{ogolne_natezenie} do następującego wyrażenia: 
	\begin{equation}\label{natezenie_liniowe}
		J_{i}=\sum_{j=1}^{n}\left[\pdv{J_{i}}{X_{j}}X_{j}\right],
	\end{equation}
	które nazywane jest równaniem fenomenologicznym. W odniesieniu do równania \eqref{natezenie_liniowe} trzeba wspomnieć, że stosuje się tutaj zasadę symetrii Curie-Prigogine'a, która mówi, że przepływy i siły termodynamiczne muszą mieć taki sam charakter tensorowy \cite{orlik}.
	% na przykład dyfuzja (wektorowy) i przepływ ciepła (wektorowy) jest dozwolony, jednak dyfuzja (wektorowy)  oraz reakcja chemiczna (skalarny) jest niedozwolony. Zasada ta może być jednak złamana w przypadku procesów nieliniowych . \par
	Zapiszmy \(\frac{\partial J_{i}}{\partial X_{j}}\) jako \(L_{ij}\)
	\begin{equation}
		J_{i}=\sum_{j=1}^{n}L_{ij}X_{j}.
	\end{equation}
	Natężenie przepływu może zależeć tylko od bodźca skoniungowanego jak w prawie Fouriera \(\left(\boldsymbol{J}_{q}=-k\nabla T\right)\), są to wtedy procesy proste \cite{orlik}. Mogą one też zależeć od innych bodźców, przykładowo efekt Seebecka oraz Peltiera: \cite{Ceynowa2008}
	\begin{equation}
	\begin{split}
		\boldsymbol{Q} &= L_{qq}\Delta T+L_{qI}\Delta \phi \\
		\boldsymbol{I} &= L_{Iq}\Delta T+L_{II}\Delta \phi
	\end{split}
	\end{equation}
	Występują w nich procesy krzyżowe; różnica temperatury wywołuje przepływ prądu oraz różnica potencjału elektrycznego wywołuje przepływ ciepła.
	Okazuje się, że współczynniki krzyżowe są sobie równe: \(L_{qI}=L_{Iq}\). Jest to reguła przemienności Onsagera, która została udowodniona doświadczalnie oraz na podstawie fizyki statystycznej. \par
	% Produkcja entropii jest wyrażona za pomocą sił termodynamicznych i przepływów jako: 
	% \begin{equation}
	% 	\sigma = \sum_{i}J_{i}X_{i} = \sum_{i}\sum_{j}^{n}L_{ij}X_{i}X_{j}
	% \end{equation}
	% co zostało wyprowadzone w Dodatku A na stronie \pageref{sec: dodatek A}. \par
	
	%Jednak w chemii termodynamika liniowa nie opisuje dobrze szybkości zachodzenia reakcji chemicznych, których prędkość zazwyczaj zależy od wyższych potęg stężenia związków. \par
	
	%Termodynamika liniowa ma zastosowanie w większości przypadków technicznych, jednak nie są one w stanie opisać reakcji chemicznych, których szybkość zależy od wyższych niż 1 potęg stężenia składników. \par
	Istnieje kilka podejść w próbie wyjścia poza zakres liniowej termodynamiki nierównowagowej. Jedno z nich zakłada, że współczynniki \(L_{ij}\) zależą od bodźców i przepływów, a więc
	\begin{align}
		& \pdv{L_{ij}}{X_{k}}\neq 0, && \pdv{L_{ij}}{J_{k}}\neq 0.
	\end{align}
	Równanie fenomenologiczne \eqref{natezenie_liniowe} zostaje zachowane, ale ten zabieg powoduje, że teoria staje się nieliniowa. \par
	Kolejne podejscie zakłada, że w rozwinięciu \eqref{ogolne_natezenie} uwzględnia większą liczbę wyrazów, przy zachowaniu niezależności współczynników rozwinięcia. Postuluje się jednocześnie, aby spełnione były relacje przemienności:
	\begin{equation}
	\begin{split}
		L_{ij} &= L_{ji} \\
		L_{ijk} &= L_{jki} = L_{kij}.
	\end{split}
	\end{equation}
	Jednak próby nie były owocne, a przemienność współczynników \(L_{ij}, L_{ijk}, \ldots\) trudna do uzasadnienia. \par
	Bardziej owocne podejście do problemu wyjścia poza liniową termodynamikę nierównowagową, głównie w kontekście oscylacyjnych reakcji chemicznych, polegało na wykorzyskiwaniu metod stosowanych z teorii układów dynamicznych. Autorzy tej koncepcji pozostawiają postulat o istnieniu równowagi lokalnej. Hipoteza ta zakłada, że cały układ możemy podzielić na mniejsze podukłady (zwane niekiedy komórkami), w których parametry termodynamiczne są ściśle zdefiniowane, tak jak to mamy w zagadnieniach równowagowych. Parametry te w innych komórkach, mogą mieć inne wartości. \par
	Zakłada się tutaj, że te komórki są na tyle małe, iż możemy przyjąć, że parametry zmieniają się w sposób ciągły. Jednak z drugiej strony trzeba przyjąć, iż nie mogą one mieć bardzo małych rozmiarów. Ich makroskopowy charakter musi być zachowany przy założeniu dodatkowym, iż w każdej z nich panuje stan wewnętrznej równowagi. \par
	
	%Zasada minimalnej produkcji entropii mówi, że dla danego układu stacjonarnego produkcja entropii ma wartość minimalną. W układzie równowagowym produkcja entropii jest równa 0. Po wprowadzeniu siły termodynamicznej układ zostaje wytrącony z równowagi. Po zaprzestaniu wymuszania nierównowagi układ dąży do stanu równowagi \cite{orlik}. \par
	%Rozwijając entropię \(S\) w szereg Taylora otrzymujemy:
	%\begin{equation}
	%	S = S^\circ + \left(\var{S}\right)^\circ + \frac{1}{2!}\left(\var^{2}S\right)^\circ + \ldots
	%\end{equation}
	%W stanie równowagi \(\left(\var{S}\right)^\circ=0\). Ograniczając szereg do wyrazu drugiego rzędu mamy:
	%\begin{equation}
	%	S - S^\circ = \frac{1}{2!}\left(\var^{2}S\right)^\circ
	%\end{equation}
	%Ze względu na maksimum entropii w stanie równowagi: \(\left(\var^{2}S\right)^\circ<0\)
	%Można powiązać tą zależność z produkcją entropii: 
	%\begin{equation}
	%	\pdv{t}\left(S - S^\circ\right) = \frac{1}{2!}\left(\var^{2}S\right)^\circ = \sum_{i=1}^{n}J_{i}X_{i}=\sigma\geq0
	%\end{equation}

	
	
	\section{Numeryczna i teoretyczna analiza modeli reakcji chemicznych}
	W tym rozdziale przeprowadzono analizę teoretyczną oraz numeryczną modeli stosowanych w opisie reakcji oscylacyjnych. Pierwszym modelem istotnym dla rozważanych zagadnień jest model Lotki podany przez niego w roku 1910. Mimo, że oryginalnie miał on zastosowanie w badaniu wielkości populacji zwierząt, a dokładnie zależności między drapieżnikami oraz ofiarami, ma on również pewne znaczenie dla reakcji chemicznych. W 1920 Lotka, a w 1931 niezależnie Volterra, zaproponowali zmodyfikowany model nazywany modelem Lotki-Volterry. Trzecim rozpatrywanym modelem jest bruskelator opracowany przez szkołę Prigogine'a w Brukseli \cite{prigogine}. Jest on analizowany w postaci uproszczonej, jak i ogólnej z reakcjami odwracalnymi. W opisie mechanizmów tych reakcji założono, że szybkości reakcji zależą jedynie od współczynników stechiometrycznych substratów, tj.:
	\begin{equation}
		v = k[A]^{a}[B]^{b}\ldots[J]^{j},
	\end{equation}
	gdy rozpatrujemy reakcje typu: 
	\begin{center}
		\ce{aA + bB + ... + jJ ->[k] kK + lL + ... zZ}
	\end{center}\par
	%\subsubsection{Model Lotki}
	Jako pierwszy przeanalizujemy model Lotki: 
	\begin{center}
		\ce{A ->[k_1] X} \\
		\ce{X + Y ->[k_2] 2Y} \\
		\ce{Y ->[k_3] produkty}.
	\end{center}
	W modelu tym przyjmujemy, że \(A\) jest stałe. Może to być osiągnięte poprzez wykorzystanie reaktora przepływowego, w którym kontroluje się dopływ składnika \(A\). W pierwszym kroku \(A\) zostaje przekształcone w \(X\), które w drugim kroku w reakcji z \(Y\) tworzy więcej składnika \(Y\). Jest to najprostszy model zawierający autokatalizę. Układ taki wymaga więc zapoczątkowania reakcji pewną ilością \(Y\). W końcowym kroku \(Y\) zostaje przekształcone w produkty końcowe. Reakcja sumaryczna w modelu Lotki ma postać:
	\begin{center}
		\ce{A -> produkty}
	\end{center}
	zaś szybkości zmiann stężeń reagentów pośrednich $X$ oraz $Y$ opisują równania: 
	\begin{equation}
	\begin{split}
		\dv{[X]}{t}=k_{1}[A]-k_{2}[X][Y] \\
		\dv{[Y]}{t}=k_{2}[X][Y]-k_{3}[Y].
	\end{split}
	\end{equation}\par
	% \subsubsection{Model Lotki-Volterry}
	Kolejnym modelem jest model Lotki-Volterry, który jest modyfikacją powyższego:
	\begin{center}
		\ce{A + X ->[k_1] 2X} \\
		\ce{X + Y ->[k_2] 2Y} \\
		\ce{Y ->[k_3] produkty}
	\end{center}
	W modelu tym zmodyfikowano pierwszy krok poprzez wprowadzenie autokatalizy. Zmienia to zachowanie się układu co zostało przeanalizowane poniżej. Konsekwencją dodania autokatalizy jest dodatkowe wprowadzenie początkowego składnika \(X\), a więc jednym z ze stanów stacjonarnych jest \(X=Y=0\). Jest to jednak rozwiązanie trywialne i układ taki jest martwy, więc nie będzie to rozpatrywane. Reakcja sumaryczna:
	\begin{center}
		\ce{A -> produkty}.
	\end{center}
	Odpowiednie szybkości zmian stężeń reagentów pośrednich mają postać:
	\begin{equation}
	\begin{split}
		\dv{[X]}{t}=k_{1}[A][X]-k_{2}[X][Y] \\
		\dv{[Y]}{t}=k_{2}[X][Y]-k_{3}[Y]
	\end{split}
	\end{equation}\par
	% \subsubsection{Model bruskelator}
	Model bruskelator ma postać:
	\begin{center}
		\ce{A ->[k_1] X} \\
		\ce{2X + Y ->[k_2] 3X} \\
		\ce{B + X ->[k_3] D + Y} \\
		\ce{X ->[k_4] E}
	\end{center}
	Pierwszy krok modelu brukselator jest taki sam jak modelu Lotki. W drugim występuje autokataliza \(Y\) do \(X\). W trzecim tworzenie \(Y\) z \(X\), natomiast w ostatnim przekształcenie \(X\) w produkty końcowe. Reakcja sumaryczna: 
	\begin{center}
		\ce{A + B -> C + D}.
	\end{center}
	Szybkości zmian stężeń reagentów pośrednich: 
	\begin{equation}\label{bruskelator_pochodne}
	\begin{split}
		\dv{[X]}{t} &= k_{1}[A]+k_{2}[X]^{2}[Y]-k_{3}[B][X]-k_{4}[X] \\
		\dv{[Y]}{t} &= -k_{2}[X]^{2}[Y]+k_{3}[B][X].
	\end{split}
	\end{equation}
	W rozdziale \ref{sec:ogolny_model_bruskelator} na stronie \pageref{sec:ogolny_model_bruskelator} będziemy analizować zmodyfikowany model bruskelatora, w którym reakcje zachodzą w obie strony. 
	
	\subsection{Metody rozwiązywania układów nieliniowych równań różniczkowych}
	W każdym z powyższych modeli otrzymujemy układy nieliniowych równań różniczkowych, które chcemy rozwiązać. 
	Układ równań liniowych pierwszego rzędu o stałych współczynnikach zawartych w macierzy $R$:
	\begin{equation*}
		\dv{X(t)}{t} = RX(t)
	\end{equation*}
	ma w ogólności rozwiązanie analityczne \cite{palczewski}
	\begin{equation*}
		X(t) = \exp(Rt)X(0)
	\end{equation*}
	Rozpatrywane układy jednak nie mają rozwiązania analitycznego i należy je rozwiązać metodami numerycznymi. \par
	W pracy tej wykorzystałem algorytmy wielokrokowe, w których jeden krok schematu numerycznego, to znaczy przejścia z punktu \(y_{n}\) do punktu \(y_{n+1}\), wykorzystuje wyniki z \(j\) kroków, gdzie \(j\leq n\). Krok jest oznaczony \(h=x_{n+1}-x_{n}\). \par
	
	Ogólna forma metody różnicowej rozwiązującej równanie różniczkowe:
	\begin{equation}\label{row_rozniczkowe}
		\dv{\vb{y}}{x}=\vb{f}(x, \vb{y})
	\end{equation}
	ma postać \cite{fortuna}:
	\begin{equation}\label{metoda_roznicowa}
		\vb{y}_{n+1} = \sum_{i=1}^{k}a_{i}\vb{y}_{n+1-i} + h\sum_{i=0}^{k}b_{i}\vb{f}(x_{n+1-i}, \vb{y}_{n+1-i}), \quad n\geq k-1.
	\end{equation}
	Wprowadzamy wielkość \(Y\), która jest dokładnym rozwiązaniem równania \eqref{row_rozniczkowe}. Możemy wtedy zapisać: 
	\begin{equation}\label{ogolny_schemat_dokladny}
		\vb{Y}_{n+1} = \sum_{i=1}^{k}a_{i}\vb{Y}_{n+1-i} + h\sum_{i=0}^{k}b_{i}\vb{f}(x_{n+1-i}, \vb{Y}_{n+1-i}) + \vb{T}_{n},
	\end{equation}
	gdzie \(\vb{T}_{n}\) to błąd metody. Równanie \eqref{ogolny_schemat_dokladny} umożliwia napisanie \(T_{n}\) w postaci:
	\begin{equation}
		\vb{T}_{n} = \vb{Y}_{n+1} - \sum_{i=1}^{k}a_{i}\vb{Y}_{n+1-i} - h\sum_{i=0}^{k}b_{i}\vb{f}(x_{n+1-i}, \vb{Y}_{n+1-i}).
	\end{equation}
	Po rozpisaniu \(Y_{n+1-i}\) w postaci szeregu Taylora wokół \(x_{n+1-k}\) otrzymujemy:
	\begin{equation}\label{blad_metody}
		\vb{T}_{n} = \sum_{j=0}^{\infty}\vb{Y}_{n+1-k}^{(j)}h^{j}\left[\frac{k^{j}}{j!} - \sum_{i=1}^{k}a_{i}\frac{(k-i)^{j}}{j!} - \sum_{i=0}^{k}b_{i}\frac{(k-i)^{j-1}}{(j-1)!}\right].
	\end{equation}
	Prawą stronę równania \eqref{blad_metody} możemy zapisać w nieco innej postaci. W tym celu zdefiniujemy wielkości \(A_{j}\) wzorami:
	\begin{equation}
	\begin{split}
		A_{0} &= 1 - \sum_{i=1}^{k}a_{i} \\
		A_{j} &= \frac{k^{j}}{j!} - \sum_{i=1}^{k}a_{i}\frac{(k-i)^{j}}{j!} - \sum_{i=0}^{k}b_{i}\frac{(k-i)^{j-1}}{(j-1)!}, j\geq 1
	\end{split}
	\end{equation}
	to współczynniki przy \(h^{j}\). 
	Korzystając z tej notacji wzór określający \(T_{n}\) ma postać:
	\begin{equation}
		\vb{T}_{n} = \sum_{j=0}^{\infty}A_{j}\vb{Y}_{n+1-k}^{(j)}h^{j}
	\end{equation}
	Jeżeli \(A_{i}=0\) dla \(i=0, 1, \ldots, p\) oraz \(A_{p+1}\neq 0\) to metoda ta jest rzędu \(p\).
	% W tabeli \ref{tab:schematy} podano wybrane wzory różnicowe wykorzystywane podczas symulacji, gdzie \(h\) jest krokiem czasowym, \(p\) to najwyższa wykładnik potęgi \(h^{s}\), przy których współczynnik wynosi \(0\), a \(A_{p+1}\) jest współczynnikiem przy kolejnej potędze. 
	W tabeli \ref{tab:schematy} podano wybrane wzory różnicowe wykorzystywane podczas symulacji. Wzory 1 - 4 są typu Adamsa-Bashfortha \cite{fortuna}. Wzory o wyższym rzędzie wymagają znajomości wartości większej ilości poprzednich kroków, więc nie mogą być one wykorzystane dla kroków początkowych. W symulacji wykorzystano progresywnie schemat 1 dla pierwszego kroku, następnie 2 dla drugiego, 3 dla trzeciego oraz 4 dla każdego kolejnego. Symbolem \(y'\) oznaczono pochodną \(y\) we wzorze \eqref{row_rozniczkowe}, za którą podstawia się wartość funkcji po prawej stronie tego równania. 
	\begin{table}[H]
		\centering
		\begin{tabular}{|l|l|c|c|}
			\hline
			Lp. & Wzór & \(p\) & \(A_{p+1}\) \\
			\hline
			1 & \(y_{n+1}=y_{n}+hy'_{n}\) & \(1\) & \(\frac{1}{2}\) \\
			\hline
			2 & \(y_{n+1}=y_{n}+\frac{h}{2}(3y'_{n}-y'_{n-1})\) & \(2\) & \(\frac{5}{12}\) \\
			\hline
			3 & \(y_{n+1}=y_{n}+\frac{h}{12}(23y'_{n}-16y'_{n-1}+5y'_{n-2})\) & \(3\) & \(\frac{3}{8}\) \\
			\hline
			4 & \(y_{n+1}=y_{n}+\frac{h}{24}(55y'_{n}-59y'_{n-1}+37y'_{n-2}-9y'_{n-3})\) & \(4\) & \(\frac{251}{720}\) \\
			\hline
			%5 & \(y_{n+1}=y_{n-3}+\frac{4h}{3}(2y'_{n}-y'_{n-1}+2y'_{n-2})\) & \(4\) & \(\frac{14}{45}\) \\
			%\hline
		\end{tabular}.
		\caption{Schematy różnicowe stosowane do rozwiązywania układów równań różniczkowych zwyczajnych}
		\label{tab:schematy}
	\end{table}
	
	
	\subsection{Stabilność rozwiązań układów równań różniczkowych}
	Na potrzeby analizy można zredukować ilość parametrów danych równań różniczkowych stosując odpowiednie podstawienia. Najpierw należy wyznaczyć współrzędne punktu stacjonarnego, w którym obie pochodne stężeń reagentów są równe zero. Analiza zostanie przedstawiona na przykładzie modelu Lotki, ale analogiczne wyprowadzenie można przeprowadzić dla każdego z tych modeli. W stanie stacjonarnym stężenia reagentów są stałe, więc \(\dv{[X]}{t}=\dv{[Y]}{t}=0\)
	\begin{equation}\label{lotka_pods}
	\begin{split}
		k_{1}[A]-k_{2}[X]_{st}[Y]_{st} &= 0 \\
		k_{2}[X]_{st}[Y]_{st}-k_{3}[Y]_{st} &= 0
	\end{split}
	\end{equation}
	Rozwiązując ten układ równań otrzymujemy
	\begin{equation}\label{lotka_stac}
	\begin{split}
		[X]_{st} &= \frac{k_{3}}{k_{2}} \\
		[Y]_{st} &= \frac{k_{1}[A]}{k_{3}}
	\end{split}
	\end{equation}
	Wprowadzamy podstawienie
	\begin{align*}
		& x=\frac{[X]}{[X]_{st}} && y=\frac{[Y]}{[Y]_{st}} && \tau=k_{3}t && a=\frac{k_{1}k_{2}[A]}{k_{3}^{2}}
	\end{align*}
	i otrzymujemy po przekształceniach dla modelu Lotki:
	\begin{equation}\label{lotka_prosty}
	\begin{split}
		\dv{x}{\tau} &= a-axy \\
		\dv{y}{\tau} &= xy-y
	\end{split}
	\end{equation}
	Analogiczne wyprowadzenie można przeprowadzić dla modelu Lotki-Volterry:
	\begin{align}\label{lotka_volterra_stac}
		& x=\frac{[X]}{[X]_{st}} && [X]_{st} = \frac{k_{3}}{k_{2}} && y=\frac{[Y]}{[Y]_{st}} && [Y]_{st} = \frac{k_{1}[A]}{k_{2}} && \tau=k_{3}t && a=\frac{k_{1}[A]}{k_{3}}.
	\end{align}
	Model Lotki-Volterry po tych przekształceniach ma postać: 
	\begin{equation}\label{lotka_volterra_prosty}
	\begin{split}
		\dv{x}{\tau} &= ax-axy \\
		\dv{y}{\tau} &= xy-y
	\end{split}
	\end{equation}
	W przypadku modelu bruskelator zastosowano następujące podstawienie:
	\begin{align*}
		& x=\frac{[X]}{[X]_{st}} && [X]_{st} = \frac{k_{1}[A]}{k_{4}} && y=\frac{[Y]}{[Y]_{st}} && [Y]_{st} = \frac{k_{3}k_{4}[B]}{k_{1}k_{2}[A]} && \tau=k_{4}t && a=\frac{k_{3}[B]}{k_{4}} && b=\frac{k_{1}^{2}k_{2}[A]^{2}}{k_{4}^{3}}
	\end{align*}
	Skutkuje to przekształceniem równania \eqref{bruskelator_pochodne} do postaci: 
	\begin{equation}\label{bruskelator_prosty}
	\begin{split}
		\dv{x}{\tau} &= 1+ax^{2}y-ax-x \\
		\dv{y}{\tau} &= -bx^{2}y+bx
	\end{split}
	\end{equation}
	Stałe w powyższych równaniach wynikają z podstawienia odpowiednich \(x\) i \(y\) do odpowiadających równań i grupowanie stałych, aby otrzymać najprostszą formę. \par
	Równania \eqref{lotka_prosty}, \eqref{lotka_volterra_prosty} oraz \eqref{bruskelator_prosty} mają stan stacjonarny w \(x=y=1\), co wynika z definicji \(x\) oraz \(y\) jako \(x=\frac{[X]}{[X]_{st}}\) oraz \(y=\frac{[Y]}{[Y]_{st}}\), które dla \([X]=[X]_{st}\) oraz \([Y]=[Y]_{st}\) są równe \(1\). Zostaną one wykorzystane do numerycznego rozwiązania równań. \par
	Na potrzeby dalszej analizy teoretycznej wprowadzam dalsze podstawienie:
	\begin{align*}
		& \gamma=x-1 && \vartheta=y-1
	\end{align*}
	To powoduje, że stan stacjonarny przesuwa się do \(\gamma=\vartheta=0\). Otrzymujemy dla modelu Lotki:
	\begin{equation}\label{lotka_0}
	\begin{split}
		\dv{\gamma}{\tau} &= -a\gamma\vartheta-a\gamma-a\vartheta \\
		\dv{\vartheta}{\tau} &= \gamma\vartheta+\gamma.
	\end{split}
	\end{equation}
	Dla modelu Lotki-Volterry:
	\begin{equation}\label{lotka_volterra_0}
	\begin{split}
		\dv{\gamma}{\tau} &= -a\gamma\vartheta-a\vartheta \\
		\dv{\vartheta}{\tau} &= \gamma\vartheta+\gamma.
	\end{split}
	\end{equation}
	Dla modelu bruskelator:
	\begin{equation}\label{bruskelator_0}
	\begin{split}
		\dv{\gamma}{\tau} &= a\gamma^{2}\vartheta+a\gamma^{2}+2a\gamma\vartheta+a\gamma+a\vartheta-\gamma \\
		\dv{\vartheta}{\tau} &= -b\gamma^{2}\vartheta-b\gamma^{2}-2b\gamma\vartheta-b\gamma-b\vartheta.
	\end{split}
	\end{equation}
	Istnienie punktu stacjonarnego nie oznacza, że jest on atraktorem. Tutaj pod pojęciem atraktora rozumiemy zbiór \(\omega\)-graniczny, którego definicja ma postać:
	\begin{definition}[Zbiór $\omega$-graniczny]
	\begin{equation*}
		\omega(p) = {y\in\mathbb{R}^{m}: y = \lim_{t\to\infty}}x(t; p),
	\end{equation*}
	gdzie \(x(t; p)\) to rozwiązanie \(\dot{x}=f(x)\) przy założeniu \(x(0; p) = p\)
	\end{definition}
	\begin{definition}[Cykl graniczny]
		"Jeśli istnieje orbita zamknięta $\gamma$, taka że dla punktów $y$ należących do pewnego otoczenia $U$ zbioru $\gamma$ mamy \(\omega(y)=\gamma\) [\ldots], to $\gamma$ nazywamy \emph{cyklem granicznym}."
	\end{definition}\noindent
	Jeśli \(\gamma=\omega(y)\) dla każdego punktu z otoczenia $U$, to $\gamma$ jest atraktorem. \cite{palczewski} \par
	Układy są badane w stanie oddalonym od stanu stacjonarnego, dlatego wybieramy taki stan jako stan odniesienia, a pozostałe jako wyprowadzone z niego zaburzeniem.
	Badanie charakteru punktu stacjonarnego układu równań różniczkowych nieliniowych jest trudne, ale można wprowadzić pewne uproszczenie i zlinearyzować ten układ \cite{kawczynski, palczewski}. Oznacza to rozwiniecie funcji po prawej stronie równań w szereg Taylora i ograniczenie go do elementu liniowego. W rezultacie przeprowadzonej operacji otrzymujemy układ równań liniowych. Działanie to jest uzasadnione tym, że badamy jedynie najbliższe otoczenie i kolejne składniki mają mniejszy wkład im bliżej punktu stacjonarnego. \par
	Po linearyzacji otrzymujemy dla modelu Lotki:
	\begin{equation}\label{lotka_lin}
	\begin{split}
		\dv{\gamma}{\tau} &= -a\gamma-a\vartheta \\
		\dv{\vartheta}{\tau} &= \gamma
	\end{split}
	\end{equation}
	Układ równań \eqref{lotka_lin} w porównaniu z \eqref{lotka_0} nie zawiera składników o całkowitej potędze większej niż 1. Odpowiednie przekształcenie dla modelu Lotki-Volterry daje układ:
	\begin{equation}\label{lotka_volterra_lin}
	\begin{split}
		\dv{\gamma}{\tau} &= -a\vartheta \\
		\dv{\vartheta}{\tau} &= \gamma,
	\end{split}
	\end{equation}
	natomiast dla modelu bruskelator:
	\begin{equation}\label{bruskelator_lin}
	\begin{split}
		\dv{\gamma}{\tau} &= (a-1)\gamma+a\vartheta \\
		\dv{\vartheta}{\tau} &= -b\gamma-b\vartheta
	\end{split}
	\end{equation}\par
	Można teraz badać stany stabilne metodami stosowanymi do analizy układów równań różniczkowych liniowych. Stabilność zależy od wartości własnych macierzy $R$ stałych oznaczonych symbolami \(\lambda_{1}\) i \(\lambda_{2}\). Odnajdujemy je rozwiązując równanie kwadratore \eqref{rown_charakterystyczne}:
	\begin{equation}\label{rown_charakterystyczne}
		a\lambda^{2} + b\lambda + c = 0.
	\end{equation}
	W tabeli \ref{tab:warunki_stabilnosci} przedstawiono zależności między pierwiastkami równania kwadratowego, a sumą i iloczynem tych pierwiastków. Suma oraz iloczyn są tutaj wykorzystywane, ponieważ można je w prosty sposów otrzymać ze wzorów Viete'a, które zostały wyprowadzone w Dodatku B na stronie \pageref{sec: dodatek B}: 
	\begin{equation}
	\begin{split}
		\lambda_{1} + \lambda_{2} &= -\frac{b}{a} \\
		\lambda_{1}\lambda_{2} &= \frac{c}{a}.
	\end{split}
	\end{equation}
	Charakter wykresu fazowego zależy od zależności między pierwiastkami równania charakterystycznego \cite{orlik}.
	Dla modelu Lotki:
	\begin{equation}\label{lotka_charakterystyczne}
	\begin{split}
		\det
		\begin{pmatrix}
			-a-\lambda & -a \\
			1 & -\lambda
		\end{pmatrix}
		=\lambda^{2}+a\lambda+a=0 \\
		\lambda_{1}+\lambda_{2}=-a \\
		\lambda_{1}\lambda_{2}=a
	\end{split}
	\end{equation}
	Dla modelu Lotki-Volterry:
	\begin{equation}\label{lotka_volterra_charakterystyczne}
	\begin{split}
		\det
		\begin{pmatrix}
			-\lambda & -a \\
			1 & -\lambda
		\end{pmatrix}
		=\lambda^{2}+a=0 \\
		\lambda_{1}+\lambda_{2}=0 \\
		\lambda_{1}\lambda_{2}=a
	\end{split}
	\end{equation}
	Dla modelu bruskelator:
	\begin{equation}\label{bruskelator_charakterystyczne}
	\begin{split}
		\det
		\begin{pmatrix}
			a-1-\lambda & a \\
			-b & -b-\lambda
		\end{pmatrix}
		=\lambda^{2}+(-a+b+1)\lambda+b=0 \\
		\lambda_{1}+\lambda_{2}=a-b-1 \\
		\lambda_{1}\lambda_{2}=b
	\end{split}
	\end{equation}
	
	\begin{table}[H]
		\centering
		\begin{tabularx}{\textwidth}{|Y|Y|Y|Y|}\cline{1-4}
			& \(\lambda_{1}+\lambda_{2}<0\) & \(\lambda_{1}+\lambda_{2}=0\) & \(\lambda_{1}+\lambda_{2}>0\) \\ \cline{1-4}
			\(\left(\frac{\lambda_{1}+\lambda_{2}}{2}\right)^{2}<\lambda_{1}\lambda_{2}\) & 
			{\begin{tabularx}{\columnwidth}{c} %11
				\(\lambda_{1}, \lambda_{2}\in\mathbb{C}\) \\
				\(\Re{\lambda_{1}}=\Re{\lambda_{2}}<0\) \\
				\(\lambda_{1}=\overline{\lambda_{2}}\) \\
				\textbf{Stabilne ognisko}
			\end{tabularx}}
			& \multirow{2}{*}{
			{\begin{tabularx}{\columnwidth}{c} %(1-2)2
				\(\lambda_{1}, \lambda_{2}\in\mathbb{C}\) \\
				\(\Re{\lambda_{1}}=\Re{\lambda_{2}}=0\) \\
				\(\lambda_{1}=-\lambda_{2}\) \\
				\textbf{Centrum}
			\end{tabularx}}
			} & 
			{\begin{tabularx}{\columnwidth}{c} %13
				\(\lambda_{1}, \lambda_{2}\in\mathbb{C}\) \\
				\(\Re{\lambda_{1}}=\Re{\lambda_{2}}>0\) \\
				\(\lambda_{1}=\overline{\lambda_{2}}\) \\
				\textbf{Niestabilne ognisko}
			\end{tabularx}} 
			\\ \cline{1-2}\cline{4-4}
			\(0<\lambda_{1}\lambda_{2}\leq\left(\frac{\lambda_{1}+\lambda_{2}}{2}\right)^{2}\) & 
			{\begin{tabularx}{\columnwidth}{c} %21
				\(\lambda_{1}, \lambda_{2}\in\mathbb{R}\) \\
				\(\lambda_{1}, \lambda_{2}<0\) \\
				\textbf{Stabilny węzeł}
			\end{tabularx}}
			& &
			{\begin{tabularx}{\columnwidth}{c} %23
				\(\lambda_{1}, \lambda_{2}\in\mathbb{R}\) \\
				\(\lambda_{1}, \lambda_{2}>0\) \\
				\textbf{Niestabilny węzeł}
			\end{tabularx}}
			\\ \cline{1-4}
			\(\lambda_{1}\lambda_{2}=0\) & 
			{\begin{tabularx}{\columnwidth}{c} %31
				\(\lambda_{1}, \lambda_{2}\in\mathbb{R}\) \\
				\(\lambda_{1}<\lambda_{2}=0\) 
			\end{tabularx}}
			& 
			{\begin{tabularx}{\columnwidth}{c} %32
				\(\lambda_{1}, \lambda_{2}\in\mathbb{R}\) \\
				\(\lambda_{1}=\lambda_{2}=0\) 
			\end{tabularx}}
			& 
			{\begin{tabularx}{\columnwidth}{c} %33
				\(\lambda_{1}, \lambda_{2}\in\mathbb{R}\) \\
				\(0=\lambda_{1}<\lambda_{2}\) 
			\end{tabularx}}
			\\ \cline{1-4}
			\(\lambda_{1}\lambda_{2}<0\) & 
			{\begin{tabularx}{\columnwidth}{c} %41
				\(\lambda_{1}, \lambda_{2}\in\mathbb{R}\) \\
				\(0<\lambda_{2}<-\lambda_{1}\) \\
				\textbf{Siodło} \\
				\textbf{(zawsze niestabilne)}
			\end{tabularx}}
			& 
			{\begin{tabularx}{\columnwidth}{c} %42
				\(\lambda_{1}, \lambda_{2}\in\mathbb{R}\) \\
				\(0<\lambda_{2}=-\lambda_{1}\) \\
				\textbf{Siodło} \\
				\textbf{(zawsze niestabilne)}
			\end{tabularx}}
			&
			{\begin{tabularx}{\columnwidth}{c} %43
				\(\lambda_{1}, \lambda_{2}\in\mathbb{R}\) \\
				\(0>\lambda_{1}>-\lambda_{2}\) \\
				\textbf{Siodło} \\
				\textbf{(zawsze niestabilne)}
			\end{tabularx}}
			\\ \cline{1-4}
		\end{tabularx}
		\caption{Warunki stabilności dla liniowego układu dwóch równań różniczkowych. W pierwszej i trzeciej kolumnie \(\lambda_{1}+\lambda_{2}\in\mathbb{R}\)}
		\label{tab:warunki_stabilnosci}
	\end{table}
	
	\begin{table}[H]
		\centering
		\begin{tabularx}{\textwidth}{|Y|Y|Y|Y|}\cline{1-4}
			& \(\lambda_{1}+\lambda_{2}<0\) & \(\lambda_{1}+\lambda_{2}=0\) & \(\lambda_{1}+\lambda_{2}>0\) \\ \cline{1-4}
			\(\left(\frac{\lambda_{1}+\lambda_{2}}{2}\right)^{2}<\lambda_{1}\lambda_{2}\) & 
			{\begin{tabularx}{\columnwidth}{c} %11
				\(0<a<4\)
			\end{tabularx}}
			& \multirow{2}{*}{
				{\begin{tabularx}{\columnwidth}{c} %(1-2)2
					-
				\end{tabularx}}
			} & 
			{\begin{tabularx}{\columnwidth}{c} %13
				-
			\end{tabularx}}
			\\ \cline{1-2}\cline{4-4}
			\(0<\lambda_{1}\lambda_{2}\leq\left(\frac{\lambda_{1}+\lambda_{2}}{2}\right)^{2}\) & 
			{\begin{tabularx}{\columnwidth}{c} %21
				\(4\leq a\)
			\end{tabularx}}
			& &
			{\begin{tabularx}{\columnwidth}{c} %23
				-
			\end{tabularx}}
			\\ \cline{1-4}
			\(\lambda_{1}\lambda_{2}=0\) & 
			{\begin{tabularx}{\columnwidth}{c} %31
				-
			\end{tabularx}}
			& 
			{\begin{tabularx}{\columnwidth}{c} %32
				\(a=0\)
			\end{tabularx}}
			& 
			{\begin{tabularx}{\columnwidth}{c} %33
				-
			\end{tabularx}}
			\\ \cline{1-4}
			\(\lambda_{1}\lambda_{2}<0\) & 
			{\begin{tabularx}{\columnwidth}{c} %41
				-
			\end{tabularx}}
			& 
			{\begin{tabularx}{\columnwidth}{c} %42
				-
			\end{tabularx}}
			&
			{\begin{tabularx}{\columnwidth}{c} %43
				\(a<0\)
			\end{tabularx}}
			\\ \cline{1-4}
		\end{tabularx}
		\caption{Warunki dla zlinearyzowanego modelu Lotki}
		\label{tab:warunki_lotka}
	\end{table}

	\begin{table}[H]
		\centering
		\begin{tabularx}{\textwidth}{|Y|Y|Y|Y|}\cline{1-4}
			& \(\lambda_{1}+\lambda_{2}<0\) & \(\lambda_{1}+\lambda_{2}=0\) & \(\lambda_{1}+\lambda_{2}>0\) \\ \cline{1-4}
			\(\left(\frac{\lambda_{1}+\lambda_{2}}{2}\right)^{2}<\lambda_{1}\lambda_{2}\) & 
			{\begin{tabularx}{\columnwidth}{c} %11
				-
			\end{tabularx}}
			& \multirow{2}{*}{
				{\begin{tabularx}{\columnwidth}{c} %(1-2)2
					\(0<a\)
				\end{tabularx}}
			} & 
			{\begin{tabularx}{\columnwidth}{c} %13
				-
			\end{tabularx}} 
			\\ \cline{1-2}\cline{4-4}
			\(0<\lambda_{1}\lambda_{2}\leq\left(\frac{\lambda_{1}+\lambda_{2}}{2}\right)^{2}\) & 
			{\begin{tabularx}{\columnwidth}{c} %21
				-
			\end{tabularx}}
			& &
			{\begin{tabularx}{\columnwidth}{c} %23
				-
			\end{tabularx}}
			\\ \cline{1-4}
			\(\lambda_{1}\lambda_{2}=0\) & 
			{\begin{tabularx}{\columnwidth}{c} %31
				-
			\end{tabularx}}
			& 
			{\begin{tabularx}{\columnwidth}{c} %32
				\(a=0\) 
			\end{tabularx}}
			& 
			{\begin{tabularx}{\columnwidth}{c} %33
				-
			\end{tabularx}}
			\\ \cline{1-4}
			\(\lambda_{1}\lambda_{2}<0\) & 
			{\begin{tabularx}{\columnwidth}{c} %41
				-
			\end{tabularx}}
			& 
			{\begin{tabularx}{\columnwidth}{c} %42
				\(a<0\)
			\end{tabularx}}
			&
			{\begin{tabularx}{\columnwidth}{c} %43
				-
			\end{tabularx}}
			\\ \cline{1-4}
		\end{tabularx}
		\caption{Warunki dla zlinearyzowanego modelu Lotki-Volterry}
		\label{tab:warunki_lotka_volterra}
	\end{table}

	\begin{table}[H]
		\centering
		\begin{tabularx}{\textwidth}{|Y|Y|Y|Y|}\cline{1-4}
			& \(\lambda_{1}+\lambda_{2}<0\) & \(\lambda_{1}+\lambda_{2}=0\) & \(\lambda_{1}+\lambda_{2}>0\) \\ \cline{1-4}
			\(\left(\frac{\lambda_{1}+\lambda_{2}}{2}\right)^{2}<\lambda_{1}\lambda_{2}\) & 
			{\begin{tabularx}{\columnwidth}{c} %11
				\(b+1-2\sqrt{b}<a<b+1\) \\
				\(0<b\)
			\end{tabularx}}
			& \multirow{2}{*}{
				{\begin{tabularx}{\columnwidth}{c} %(1-2)2
					 \(a=b+1\) \\
					 \(0<b\)
				\end{tabularx}}
			} & 
			{\begin{tabularx}{\columnwidth}{c} %13
				\(b+1<a<b+1+2\sqrt{b}\) \\
				\(0<b\)
			\end{tabularx}} 
			\\ \cline{1-2}\cline{4-4}
			\(0<\lambda_{1}\lambda_{2}\leq\left(\frac{\lambda_{1}+\lambda_{2}}{2}\right)^{2}\) & 
			{\begin{tabularx}{\columnwidth}{c} %21
				\(a\leq b+1-2\sqrt{b}\) \\
				\(0<b\)
			\end{tabularx}}
			& &
			{\begin{tabularx}{\columnwidth}{c} %23
				\(b+1+2\sqrt{b}\leq a\) \\
				\(0<b\)
			\end{tabularx}}
			\\ \cline{1-4}
			\(\lambda_{1}\lambda_{2}=0\) & 
			{\begin{tabularx}{\columnwidth}{c} %31
				\(a<1\) \\
				\(b=0\)
			\end{tabularx}}
			& 
			{\begin{tabularx}{\columnwidth}{c} %32
				\(a=1\) \\
				\(b=0\) 
			\end{tabularx}}
			& 
			{\begin{tabularx}{\columnwidth}{c} %33
				\(1<a\) \\
				\(b=0\)
			\end{tabularx}}
			\\ \cline{1-4}
			\(\lambda_{1}\lambda_{2}<0\) & 
			{\begin{tabularx}{\columnwidth}{c} %41
				\(a<b+1\) \\
				\(b<0\)
			\end{tabularx}}
			& 
			{\begin{tabularx}{\columnwidth}{c} %42
				\(a=b+1\) \\
				\(b<0\)
			\end{tabularx}}
			&
			{\begin{tabularx}{\columnwidth}{c} %43
				\(b+1<a\) \\
				\(b<0\)
			\end{tabularx}}
			\\ \cline{1-4}
		\end{tabularx}
		\caption{Warunki dla zlinearyzowanego modelu bruskelator}
		\label{tab:warunki_bruskelator}
	\end{table}
	W tabelach \ref{tab:warunki_lotka}, \ref{tab:warunki_lotka_volterra} oraz \ref{tab:warunki_bruskelator} zebrano warunki z tabeli \ref{tab:warunki_stabilnosci}, które mają znaczenie dla analizy równań \eqref{lotka_charakterystyczne}, \eqref{lotka_volterra_charakterystyczne} oraz \eqref{bruskelator_charakterystyczne}. Niektóre warunki nie są możliwe do spełnienia dla parametrów rzeczywistych, taka komórka zawiera '-'. \par
	Przejdziemy teraz do badania zachowania trajektorii w przestrzeniach fazowych rozważanych modeli. Do wygenerowania pól wektorowych i trajektorii na rysunkach odpowiednio \ref{fig:stabilne_ognisko_lotka}, \ref{fig:stabilny_wezel_lotka}, \ref{fig:stabilne_ognisko_lotka_volterra}, \ref{fig:stabilny_wezel}, \ref{fig:stabilne_ognisko} oraz \ref{fig:niestabilne_ognisko} wykorzystano równania przed linearyzacją \eqref{lotka_prosty}, \eqref{lotka_volterra_prosty} oraz \eqref{bruskelator_prosty}. Wektory na wykresach są znormalizowane. Interesuje nas jedynie ich kierunek, a nie sama wartość. \par
	Na rysunkach \ref{fig:stabilne_ognisko_lotka} oraz \ref{fig:stabilny_wezel_lotka} przedstawiają typowe wykresy dla modelu Lotki. Należy zwrócić uwagę na fakt, że zawsze jest to tor zbiegający do punktu stacjonalnego. 
	\begin{figure}[H]
		\centering
		\includegraphics[scale=0.8]{"lotka3; a=0.1".png}
		\caption{Stabilne ognisko; Model Lotki, a=0.1}
		\label{fig:stabilne_ognisko_lotka}
	\end{figure}
	\begin{figure}[H]
		\centering
		\includegraphics[scale=0.8]{"lotka; a=5".png}
		\caption{Stabilny węzeł; Model Lotki, a=5}
		\label{fig:stabilny_wezel_lotka}
	\end{figure}\newpage
	Na rysunku \ref{fig:stabilne_ognisko_lotka_volterra} pokazano typowy wykres dla modelu Lotki-Volterry, w którym \(a=1\). Jest to zawsze krzywa zamknięta niezależnie od wyboru warunków początkowych. 
	\begin{figure}[H]
			\centering
			\includegraphics[scale=0.8]{"lotka_volterra; a=1".png}
			\caption{Stabilne ognisko; Model Lotki-Volterry, a=1}
			\label{fig:stabilne_ognisko_lotka_volterra}
		\end{figure}
	Wykresy fazowe na rysunkach \ref{fig:stabilny_wezel}, \ref{fig:stabilne_ognisko} oraz \ref{fig:niestabilne_ognisko} przedstawiają przykładowe trajektorie dla najważniejszych przypadków modelu brukselator. Wykorzystano odpowiednio stałe: \(a=0,5; b=4\), \(a=3; b=4\) oraz \(a=7; b=4\). Parametry te dobrano tak, aby należały one do przedziałów podanych w tabeli \ref{tab:warunki_bruskelator}. \par
	\begin{figure}[H]
		\centering
		\includegraphics[scale=0.8]{"brusselator; a=0.5; b=4".png}
		\caption{Stabilny węzeł; Model bruskelator, a=0.5, b=4}
		\label{fig:stabilny_wezel}
	\end{figure}
	\begin{figure}[H]
		\centering
		\includegraphics[scale=0.8]{"brusselator; a=3; b=4".png}
		\caption{Stabilne ognisko; Model bruskelator, a=3, b=4}
		\label{fig:stabilne_ognisko}
	\end{figure}
	\begin{figure}[H]
		\centering
		\includegraphics[scale=0.8]{"brusselator; a=7; b=4".png}
		\caption{Niestabilne ognisko; Model bruskelator, a=7, b=4}
		\label{fig:niestabilne_ognisko}
	\end{figure}
	Trajektorie zachowują się w sposób oczekiwany według tabeli \ref{tab:warunki_bruskelator} i literatury, co dowodzi poprawnego działania stworzonego programu, którego kod znajduje się w dodatku C na stronie \pageref{sec: dodatek C} \cite{orlik}. Na rysunku \ref{fig:niestabilne_ognisko} zachodzi jednak coś na pierwszy rzut oka sprzecznego z analizą układu zlinearyzowanego według warunku "Niestabilne ognisko" w tabeli \ref{tab:warunki_stabilnosci}. Układ osiąga stabilny cykl graniczny. Jest to niemożliwe w przypadku układów liniowych do których je sprowadziliśmy poprzez linearyzację, jednak linearyzacja jest dobrym przybliżeniem jedynie w najbliższym otoczeniu punktu stacjonarnego, cykl graniczny widoczny na rysunku \ref{fig:niestabilne_ognisko} jest więc przejawem nieliniowości układu równań modelu bruskelator \cite{orlik}. Istnieje na jednak twierdzenie, które wyjaśnia to zachowanie, jest to twierdzenie Poincar\'{e}go-Bendixsona. \par
	\begin{theorem}[\textbf{Poincar\'{e}go-Bendixsona}]
		"Jeśli w przestrzeni fazowej będącej podzbiorem płaszczyzny \(\mathbb{R}^{2}\) orbita zawiera co najmniej jeden swój punkt graniczny, to jest ona punktem krytycznym albo orbitą zamkniętą" \cite{palczewski}
	\end{theorem}
	Z twierdzenia tego możemy wywnioskować, że punkt krytyczny, zwany również punktem stacjonarnym, jest jedynym punktem zbioru granicznego, co ma miejsce w przypadku wykresu fazowego typu stabilne ognisko, albo istnieje cykl graniczny, co można zaobserwować na wykresach odpowiadającym modelowi klasycznego bruskelatora oraz bruskelatora z reakcjami odwracalnymi. Model ten został opisany poniżej. \par
	Produkcja entropii powiązana jest z przebiegiem reakcji chemicznych zależnością:
	\begin{equation}\label{de_donder}
		T\dd_{i}S=A\dd{\xi}
	\end{equation}
	lub dla wielu reakcji w postaci bardziej uogólnionej: 
	\begin{equation}\label{de_donder_ogolny}
		T\dd_{i}S=\sum_{r}A_{r}\dd{\xi_{r}}.
	\end{equation}
	%Nie można jednak wykorzystać tych modeli do ilościowego symulowania produkcji entropii, ponieważ nie można zdefiniować \(A\) dla reakcji, które nie mają stanu równowagi. Wynika to z zapisu powinowactwa chemicznego \(A\) w roztworze:
	%\begin{equation}
	%	A = RT\ln(K) - RT\ln(\prod_{i}c_{i}^{\nu_{i}}), 
	%\end{equation}
	%gdzie \(K\) to stała równowagi reakcji, \(R\) to uniwersalna stała gazowa, \(T\) - temperatura bezwzględna, \(c_{i}\) - stężenie \(i\)-tego składnika, a \(\nu_{i}\) to współczynnik stechiometryczny reagenta \(i\) (dodatni dla produktów po prawej stronie, ujemny dla substratów po lewej). W kinetyce chemicznej stała równowagi jest równa \(K = \frac{k_{1}}{k_{-1}}\) dla reakcji \ce{X <=>[k_{1}][K_{-1}] Y}. W przypadku rozważanych powyżej reakcji \(k_{-1}=0\), występuje więc dzielenie przez \(0\). Można jedynie stwierdzić, że w granicy \(A\) jest dodatnie
	W przypadku opisanych powyżej reakcji zakładamy, że zachodzą one tylko w stronę produktów. W takim przypadku powinowactwo chemiczne $A_{r}$ jest dodanie dla każdych stężeń reagentów \(c_{i}>0\). Stąd jako że reakcja przebiega w powyższych przykładach w stronę prawą to \(\dd{\xi}>0\), a więc ze wzoru \eqref{de_donder_ogolny} można stwierdzić, że:
	\begin{equation}
		\dd_{i}S=\frac{1}{T}\sum_{r}A_{r}\dd{\xi_{r}}>0.
	\end{equation}
	% 
	% \begin{equation}\label{de donder}
	% 	T\dd_{i}S=A\dd{\xi}
	% \end{equation}
	% lub dla wielu reakcji w postaci bardziej uogólnionej: 
	% \begin{equation}\label{de donder ogolny}
	% 	T\dd_{i}S=\sum_{r}A_{r}\dd{\xi_{r}}.
	% \end{equation}
	% W przypadku reackcji prostych, na przykład \ce{X -> Y}, nie jest możliwe zdefiniowane powinowactwa chemicznego,ponieważ należy narzucić warunek \(A=0\) w stanie równowagi, a taki stan nie występuje w przypadku tego typu reakcji. Możliwe jest jedynie stwierdzenie, że skoro dana reakcja zachodzi, to \(A_{i}>0\) oraz \(\dd{\xi_{i}}>0\). Temperatura bezwzględna jest zawsze dodatnia, stąd \(\dd_{i}S>0\). \par
	
	\subsection{Ogólny model bruskelator}\label{sec:ogolny_model_bruskelator}
	Będziemy teraz analizować uogólniony model bruskelatora, w którym reakcje mogą przebiegać w dwie strony z różnymi stałymi szybkości reakcji. Poprzedni model jest szczególnym przypadkiem poniższego przy założeniu \(k_{-1}=k_{-2}=k_{-3}=k_{-4}=0\). 
	Uogólniona forma modelu bruskelator ma postać:
	\begin{center}
		1: \ce{A <=>[k_{1}][k_{-1}] X} \\
		2: \ce{2X + Y <=>[k_{2}][k_{-2}] 3X} \\
		3: \ce{B + X <=>[k_{3}][k_{-3}] D + Y} \\
		4: \ce{X <=>[k_{4}][k_{-4}] E}.
	\end{center}
	Powyższe reakcje z lewej strony oznaczono symbolami '1:', '2:', '3:' oraz '4:', aby ponumerować etapy rozpatrywanej reakcji. \newline
	Reakcje są odwracalne i przebiegają przy różncyh stałych prędkości reakcji oznaczonych \(k_{i}\) oraz \(k_{-i}\) dla rekacji odpowiednio w prawą i lewą stronę. \par
	Całkowita zmiana reagentów \(X\) oraz \(Y\) ma postać:
	\begin{align}
		\dv{[X]}{t} &= k_{1}[A] + k_{2}[X]^{2}[Y] - k_{3}[B][X] - k_{4}[X] - k_{-1}[X] - k_{-2}[X]^{3} + k_{-3}[D][Y] + k_{-4}[E] \\
		\dv{[Y]}{t} &= - k_{2}[X]^{2}[Y] + k_{3}[B][X] + k_{-2}[X]^{3} - k_{-3}[D][Y]
	\end{align}
	Rozdzielam przyrosty na dwie części, odpowiadające reakcjom w prawą oraz lewą stronę:
	\begin{align}
		\dv{[X]_{1}}{t} &= k_{1}[A] + k_{2}[X]^{2}[Y] - k_{3}[B][X] - k_{4}[X] \\
		\dv{[Y]_{1}}{t} &= - k_{2}[X]^{2}[Y] + k_{3}[B][X] \\
		\dv{[X]_{2}}{t} &= - k_{-1}[X] - k_{-2}[X]^{3} + k_{-3}[D][Y] + k_{-4}[E] \\
		\dv{[Y]_{2}}{t} &= + k_{-2}[X]^{3} - k_{-3}[D][Y]
	\end{align}
	Stany stacjonarne odpowiadające odpowiednio \([X]_{1}\) i \([Y]_{1}\) oraz \([X]_{2}\) i \([Y]_{2}\) to:
	\begin{align}
		[X]_{st, 1} &= \frac{k_{1}[A]}{k_{4}} \\
		[X]_{st, 2} &= \frac{k_{-4}[E]}{k_{-1}} \\
		[Y]_{st, 1} &= \frac{k_{3}k_{4}[B]}{k_{1}k_{2}[A]} \\
		[Y]_{st, 2} &= \frac{k_{-2}k_{-4}^{3}[E]^{3}}{k_{-1}^{3}k_{-3}[D]}
	\end{align}
	Przyjmuję, że mogę dowolnie kontrolować stężenia reagentów \([A]\), \([B]\), \([D]\) i \([E]\). \([A]\) oraz \([B]\) pozostają dowolnymi parametrami, natomiast \([D]\) i \([E]\) są zależne od innych parametrów. Po przyrównaniu \([X]_{st, 1}\) oraz \([X]_{st, 2}\) i analogicznie dla \([Y]\) otrzymujemy wartości dla \([D]\) oraz \([E]\):
	\begin{align}
		[D] &= \frac{k_{1}^{4}k_{2}k_{-2}[A]^{4}}{k_{3}k_{-3}k_{4}^{4}[B]} \\
		[E] &= \frac{k_{1}k_{-1}[A]}{k_{4}k_{-4}}
	\end{align}
	Wspólna wartość stężeń dla stanu stacjonarnego:
	\begin{align}
		[X]_{st} &= \frac{k_{1}[A]}{k_{4}} \\
		[Y]_{st} &= \frac{k_{3}k_{4}[B]}{k_{1}k_{2}[A]}
	\end{align}
	Dla zwiększenia przejrzystości równań wprowadzam oznaczenia: 
	\begin{align}
		[X] &= x[X]_{st} = x\frac{k_{1}[A]}{k_{4}} \\
		[Y] &= y[Y]_{st} = y\frac{k_{3}k_{4}[B]}{k_{1}k_{2}[A]} \\
		\tau &= k_{4}t \\
		a &= \frac{k_{3}[B]}{k_{4}} \\
		b &= \frac{k_{1}^{2}k_{2}[A]^{2}}{k_{4}^{3}} \\
		c &= \frac{k_{-1}}{k_{4}} \\
		d &= \frac{k_{1}^{4}k_{2}k_{-2}[A]^{4}}{k_{3}k_{4}^{5}[B]}
	\end{align}
	Równania różniczkowe mają wtedy postać
	\begin{equation}
	\begin{split}
		\dv{x}{\tau} &= 1 + ax^{2}y - ax - x - cx - bc^{3} + by + c \\
		\dv{y}{\tau} &= -bx^{2}y + bx + dx^{3} - dy, 
	\end{split}
	\end{equation}
	a punkt stacjonarny występuje dla \(x=1, y=1\). 
	Po wprowadzeniu podstawienia:
	\begin{gather*}
		\gamma = x - 1 \\
		\vartheta = y - 1
	\end{gather*}
	i linearyzacji otrzymujemy:
	\begin{equation}\label{brus_rev_lin}
	\begin{split}
		\dv{\gamma}{\tau} &= (a - c - 3b - 1)\gamma + (a + b)\vartheta \\
		\dv{\vartheta}{\tau} &= (- b + 3d)\gamma + (- b - d)\vartheta.
	\end{split}
	\end{equation}
	% do usuniecia?
	Równanie charakterystyczne dla układu \eqref{brus_rev_lin}:
	\begin{equation}
		\lambda^{2} - (a - c - 4b - d - 1)\lambda + (-4ad + bc + cd + 4b^{2} + b + d).
	\end{equation}\par
	Powinowactwo chemiczne w stanie równowagi każdego z równań z osobna wynosi \(A=0\) \cite{pigon1}. W ogólnej postaci ma ono postać:
	\begin{equation}
		A = A_{0} - RT\ln(\prod_{i}c_{i}^{\nu_{i}}), 
	\end{equation}
	\(RT\) jest jedynie stałą i na potrzeby symulacji przyjąłem \(RT=1\). Otrzymujemy dla każdej z reakcji odpowiednio:
	\begin{align}
		1: & A_{1} = \ln(\frac{1}{cx}) \\
		2: & A_{2} = \ln(\frac{by}{dx}) \\
		3: & A_{3} = \ln(\frac{bx}{dy}) \\
		4: & A_{4} = \ln(\frac{x}{c}).
	\end{align}
	Liczba postępu reakcji wyrażona jest równością: 
	\begin{equation}
		\dd{\xi} = \frac{\dd{n_{i}}}{\nu_{i}}
	\end{equation}
	dla dowolnego reagenta, lub używając \(\dd{c_{i}} = \frac{\dd{n_{i}}}{V}\), gdzie \(V\) jest objętością, która także mogę przyjąc, że jest równa \(V=1\). Otrzymane liczby postępu reakcji dla poszczególnych reakcji:
	\begin{align}
		\dv{\xi_{1}}{\tau} &= [X]_{st}(1 - cx) \\
		\dv{\xi_{2}}{\tau} &= [X]_{st}(ax^{2}y - \frac{ad}{b}x^{3}) \\
		\dv{\xi_{3}}{\tau} &= [X]_{st}(ac - \frac{ad}{b}y) \\
		\dv{\xi_{4}}{\tau} &= [X]_{st}(x - c).
	\end{align}
	\([X]_{st}\) można oczywiście przyjąć, że jest równe \([X]_{st}=1\). Z prawa de Dondera \(T\dd_{i}S=\sum_{r}A_{r}\xi_{r}\) przyjmując \(T=1\) otrzymujemy
	\begin{equation}
		\dv{_{i}S}{\tau} = \ln(\frac{1}{cx})(1 - cx) + \ln(\frac{by}{dx})(ax^{2}y - \frac{ad}{b}x^{3}) + \ln(\frac{bx}{dy})(ac - \frac{ad}{b}y) + \ln(\frac{x}{c})(x - c)
	\end{equation}
	Założenia \(R=T=V=[X]_{st}=1\) uargumentowane są tym, że interesuje nas jedynie charakter zmienności entropii w czasie, a nie konkretna wartość entropii. Jest to jedynie model, który nie odpowiada żadnemu rzeczywistemu układowi. Oczywiście wprowadzenie takich założeń zmienia jednostkę entropii, jednak ważna dla nas jest jedynie wartość i możemy ten fakt pominąć. \par
	%Wykresy otrzymane z przeprowadzonej symulacji dla kroku symulacji \(h = \dd{\tau} = 0,001\) oraz warunku początkowego \(x=1, y=2\):
	Na rysunku \ref{fig:rev_brus_fazowy} przedstawiony jest wykres fazowy \(y(x)\). Można na nim zaobserwować osiągniecie cyklu granicznego dla stanu początkowego \(x=1, y=2\), który jest poza tym cyklem. Na rysunku \ref{fig:rev_brus_xy_t} pokazano zależności \(x(\tau)\) linią ciagłą oraz \(y(\tau)\) linią przerywaną. Wyraźniej widać tu cykliczne zmiany tych stężeń. Ostatni rysunek \ref{fig:rev_brus_entropia} przedstawia entropię \(S(\tau)\) linią ciagłą oraz szybkość jej przyrostu \(\dv{S}{\tau}(\tau)\) linią przerywaną. Ważną cechą jest fakt, że \(\dv{S}{\tau}(\tau)>0\) dla dowolnego \(\tau\), co pokazuje, że nawet dla struktur dyssypatywnych entropia stale rośnie. Błędna interpretacja drugiej zasady termodynamiki sugerowała by jednak jej spadek, ponieważ układ taki jest w pewnym sensie uporządkowany \cite{orlik_sily_w_przyrodzie}.
	\begin{figure}[H]
		\centering
		\includegraphics[scale=0.8]{"brusselator_rev1; a=9, b=1, c=1, d=0.1".png}
		\caption{Wykres fazowy dla a=9, b=1, c=1, d=0,1}
		\label{fig:rev_brus_fazowy}
	\end{figure}
	\begin{figure}[H]
		\centering
		\includegraphics[scale=0.8]{"brusselator_rev2; a=9, b=1, c=1, d=0.1".png}
		\caption{Zależność wielkości \(x\) oraz \(y\) od \(\tau\)}
		\label{fig:rev_brus_xy_t}
	\end{figure}
	\begin{figure}[H]
		\centering
		\includegraphics[scale=0.8]{"brusselator_rev3; a=9, b=1, c=1, d=0.1".png}
		\caption{Zależność wielkości \(S\) oraz \(\dv{S}{\tau}\) od \(\tau\)}
		\label{fig:rev_brus_entropia}
	\end{figure}
	
	\subsection{Porównanie metod numerycznych}
	W tej sekcji porównamy wykresy otrzymane przy tym samym kroku, ale innych metodach. Jeden z rezultatów tego porównania został zaprezentowany na rysunku \ref{lotka_volterra_porownanie_1000}.
	\begin{figure}
		\centering
		\includegraphics[scale=0.8]{"lotka_volterra_porownanie_1000".png}
		\caption{Porównanie w modelu Lotki-Volterry}
		\label{lotka_volterra_porownanie_1000}
	\end{figure}
	Wykres w kolorze czerwonym odpowiada wykorzystywanej metodzie opisanej wcześniej, natomiast w kolorze zielonym otrzymano wykorzystując jedynie metodę rzędu pierwszego, zwaną metodą Newtona. Wykorzystano model Lotki-Volterry dla \(a=1\) oraz krok czasowy \(h=0,001\) przez \(1000000\) kroków. Można zauważyć, że rozwiązanie metodą przeze mnie wykorzystywaną jest zbieżne przez długi czas, natomiast prostsza metoda rozbiega się po czasie. \par
	%Analogiczne rozważania dla modelu bruskelator daje następujące wykresy:
	Analogiczne porównanie przeprowadzono dla modelu bruskelatora, a przykładowy rezultat dla wybranych warunków początkowych został zaprezentowany na rysunku \ref{bruskelator_porownanie_1000}. Krok czasowy oraz ich ilość jest taka sama jak w poprzednim przypadku dla modulu Lotki-Volterry. Za model posłużył model bruskelator z reakcjami zachodzącymi tylko w jedną stronę przy parametrach \(a=7\) oraz \(b=4\).
	
	\begin{figure}
		\centering
		\includegraphics[scale=0.8]{"bruskelator_porownanie_1000".png}
		\caption{Porównanie w modelu bruskelator}
		\label{bruskelator_porownanie_1000}
	\end{figure}
	Możemy zaobserwować, że w tym przypadku obie metody są zbieżne. Jest to spowodowane tym, że cykl graniczy jest w modelu bruskelator atraktorem, podczas gdy w modely Lotki-Volterry nim nie jest. Sprawia to, że odstępstwa od cyklu są korygowane, aby znowu ten cykl osiagnąć. \par
	
	
	\section{Podsumowanie}
	W pracy tej przedstawiono podstawy teorii termodynamiki nierównowagowej. Są to pojęcia produkcji entropii, hipotezy lokalnej równowagi. Wykorzystano pewne zagadnienia chemiczne, na przykład mechanizmu reakcji chemicznej oraz jej szybkości z działu kinetyki chemicznej. Wykorzystano popularne modele reakcji chemicznych do ich symulacji, są to modele Lotki, Lotki-Volterry oraz bruskelatora. Miały one duże znaczenie historyczne. Do ich analizy wykorzystano narzędzia matematyczne układów autonomicznych oraz metody numeryczne. \par
	W wyniku tych symulacji otrzymano wykresy fazowe przedstawiające oscylacje reagentów przejściowych. Pokazuje to, że według modeli reakcje oscylacyjne są możliwe, co jest potwierdzone przez rzeczywiscie reakcje w układach homogenicznych. Ważniejszymi przykładami są reakcje Bielousova-Żabotyńskiego oraz Briggsa-Rauschera.
	
	%\blindtext
	%\section{Sekcja 2}
	%\blindtext
	%\section{Sekcja 3}
	%\blindtext
	%\section{Sekcja 4}
	%\blindtext 
	\pagebreak
	%\section*{Wykaz literatury}
	\addcontentsline{toc}{section}{Wykaz literatury}
	\printbibliography[title=Wykaz literatury]
	\pagebreak
	%\section*{Wykaz rysunków}
	\addcontentsline{toc}{section}{Wykaz rysunków}
	\listoffigures
	\pagebreak
	%\section*{Wykaz tabel}
	\addcontentsline{toc}{section}{Wykaz tabel}
	\listoftables
	\pagebreak
	
	\section*{Dodatek A}\label{sec: dodatek A}
	\addcontentsline{toc}{section}{Dodatek A}
	W tym dodatku zamieszczone zostały przykładowe reakcje oraz odpowiednie im prędkości reakcji chemicznej. 
	\begin{align*}
		&\ce{2N2O2 -> 4NO2 + O2} &&v=k\left[\ce{N2O2}\right] \\
		&\ce{CH3COCH3 + I2 -> CH3COCH2I + HI} &&v=k\left[\ce{CH3COCH3}\right] \\
		&\ce{H2 + Br2 -> 2HBr} &&v=\frac{k_{1}\left[\ce{H2}\right]\left[\ce{Br2}\right]^{\sfrac{1}{2}}}{1+k_{2}\sfrac{\left[\ce{HBr}\right]}{\left[\ce{Br2}\right]}}.
	\end{align*}
	Każdą z tych reakcji można rozdzielić na szereg występujących jednocześnie reakcji elementarnych. Przykładowo dla syntezy bromowodoru z cząsteczkowego wodoru i bromu:
	\begin{align*}
		&\ce{H2 + Br2 -> 2HBr} &&\text{reakcja sumaryczna} \\
		&\ce{Br2 -> 2Br^{.}} &&\text{reakcja elementarna} \\
		&\ce{Br^{.} + H2 -> HBr + H^{.}} &&\text{reakcja elementarna} \\
		&\ce{H^{.} + Br2 -> HBr + Br^{.}} &&\text{reakcja elementarna}.
	\end{align*}
	Jednak analogiczna reakcja syntezy jodowodoru przebiega w sposób bezpośredni: 
	\begin{align*}
		&\ce{H2 + I2 -> 2HI}
	\end{align*}
	co oznacza, że każdą reakcję należy rozpatrywać osobno i nie ma jednego uniwersalnego schematu \cite{pigon1}.
	
	\section*{Dodatek B}\label{sec: dodatek B}
	\addcontentsline{toc}{section}{Dodatek B}
	Równanie kwadratowe można przedstawić w dwóch formach:
	\begin{equation}\label{dodB:kwad_normalne}
		a\lambda^{2} + b\lambda + c = 0
	\end{equation}
	oraz
	\begin{equation}\label{dodB:kwad_pierwiastki}
		a(\lambda - \lambda_{1})(\lambda - \lambda_{2}) = 0,
	\end{equation}
	gdzie \(\lambda_{1}\) oraz \(\lambda_{2}\) to pierwiastki tego równania. \par
	Po rozwinięciu równania \eqref{dodB:kwad_pierwiastki} otrzymujemy: 
	\begin{equation}
		a\lambda^{2} - a(\lambda_{1} + \lambda_{2})\lambda + a\lambda_{1}\lambda_{2} = 0.
	\end{equation}
	Przyrównując tak otrzymane równanie do \eqref{dodB:kwad_normalne}: 
	\begin{equation}
		a\lambda^{2} + b\lambda + c = a\lambda^{2} - a(\lambda_{1} + \lambda_{2})\lambda + a\lambda_{1}\lambda_{2}.
	\end{equation}
	Aby było to spełnione współczynniki przy tych samych potęgach \(\lambda\) muszą być sobie równe. Z takiego warunku otrzymujemy:
	\begin{equation}
	\begin{split}
		a &= a \\
		b &= - a(\lambda_{1} + \lambda_{2}) \\
		c &= a\lambda_{1}\lambda_{2},
	\end{split}
	\end{equation}
	co po przekształceniach daje na wzory Vi\'{e}te'a:
	\begin{equation}
	\begin{split}
		\lambda_{1} + \lambda_{2} &= -\frac{b}{a} \\
		\lambda_{1}\lambda_{2} &= \frac{c}{a}
	\end{split}
	\end{equation}
	% \section*{Dodatek A}\label{sec: dodatek A}
	% \addcontentsline{toc}{section}{Dodatek A}
	% \begin{gather}
	% 	\dd{u}=T\dd{s}-p\dd{V}+\sum_{i}\mu_{i}\dd{n_{i}}; \dd{V}=0 \label{dod A: 1} \\
	% 	\dd{s}=\frac{1}{T}\dd{u}-\sum_{i}\frac{\mu_{i}}{T}\dd{n_{i}} \label{dod A: 2} \\
	% 	\dv{s}{t}=\frac{1}{T}\dv{u}{t}-\sum_{i}\frac{\mu_{i}}{T}\dv{n_{i}}{t} \label{dod A: 3} \\
	% 	\vb{J}_{s}=\frac{1}{T}\vb{J}_{u}-\sum_{i}\frac{\mu_{i}}{T}\vb{J}_{i} \label{dod A: 4}
	% \end{gather}
	% Równania ciągłości:
	% \begin{gather}
	% 	\dv{s}{t}=-\div{\vb{J}_{s}}+\sigma \label{dod A: 5} \\
	% 	\dv{u}{t}=-\div{\vb{J}_{u}} \label{dod A: 6} \\
	% 	\dv{n_{i}}{t}=-\div{\vb{J}_{i}}+\dv{n_{i;reak}}{t}; \dd{n_{i;reak}}=\sum_{r}\nu_{ir}\dd{\xi_{r}} \label{dod A: 7} \\
	% 	\dv{n_{i}}{t}=-\div{\vb{J}_{i}}+\sum_{r}\nu_{ir}\dv{\xi_{r}}{t} \label{dod A: 8}
	% \end{gather}
	% Podstawiając \eqref{dod A: 4}, \eqref{dod A: 6} oraz \eqref{dod A: 8} do \eqref{dod A: 5} otrzymujemy: 
	% \begin{equation} \label{dod A: 9}
	% 	\dv{s}{t}=\frac{1}{T}\dv{u}{t}-\sum_{i}\frac{\mu_{i}}{T}\dv{n_{i}}{t}-\left[\vb{J}_{u}\vdot\grad{\frac{1}{T}}-\sum_{i}\vb{J}_{i}\cdot\grad{\frac{\mu_{i}}{T}}-\frac{1}{T}\sum_{r}\sum_{i}\nu_{ir}\mu_{i}\dv{\xi_{r}}{t}\right]+\sigma
	% \end{equation}
	% Wprowadzamy pojęcie powinowactwa chemicznego: 
	% \begin{equation}
	% 	A_{r}=-\sum_{i}\nu_{ir}\mu_{i} \\
	% \end{equation}
	% Równanie \eqref{dod A: 9} przybiera postać:
	% \begin{equation}
	% 	\dv{s}{t}=\frac{1}{T}\dv{u}{t}-\sum_{i}\frac{\mu_{i}}{T}\dv{n_{i}}{t}-\left[\vb{J}_{u}\vdot\grad{\left(\frac{1}{T}\right)}-\sum_{i}\vb{J}_{i}\cdot\grad{\left(\frac{\mu_{i}}{T}\right)}+\sum_{r}\frac{A_{r}}{T}\dv{\xi_{r}}{t}\right]+\sigma
	% \end{equation}
	% Otrzymujemy z porównania tego wzoru z \eqref{dod A: 3}:
	% \begin{equation}
	% 	\sigma=\vb{J}_{u}\vdot\grad{\left(\frac{1}{T}\right)}-\sum_{i}\vb{J}_{i}\cdot\grad{\left(\frac{\mu_{i}}{T}\right)}+\sum_{r}\frac{A_{r}}{T}\dv{\xi_{r}}{t}
	% \end{equation}
	% Z analogicznego wyprowadzenia dla entalpii i założenia stałego ciśnienia:
	% \begin{equation}
	% 	\sigma=\vb{J}_{h}\vdot\grad{\left(\frac{1}{T}\right)}-\sum_{i}\vb{J}_{i}\cdot\grad{\left(\frac{\mu_{i}}{T}\right)}+\sum_{r}\frac{A_{r}}{T}\dv{\xi_{r}}{t}
	% \end{equation}
	% \begin{table}[H]
	% \centering
	% \begin{tabular}{|l|c|c|}
	% 	\hline
	% 	Proces & Przepływ & Siła termodynamiczna \\
	% 	\hline
	% 	Transport energii & \(\vb{J}_u\) & \(\grad{\left(\frac{1}{T}\right)}\) \\
	% 	Transport entalpii & \(\vb{J}_h\) & \(\grad{\left(\frac{1}{T}\right)}\) \\
	% 	Dyfuzja & \(\vb{J}_i\) & \(-\grad{\left(\frac{\mu_{i}}{T}\right)}\) \\
	% 	Reakcja chemiczna & \(J_{r}=\dv{\xi_{r}}{t}\) & \(\frac{A_r}{T}\) \\
	% 	\hline
	% \end{tabular}
	% \end{table}
	% \pagebreak
	%\section{Dodatek B}
	%\begin{gather}
	%	U=U\left(S, V, \xi\right) \\
	%	H=H\left(S, p, \xi\right) \\
	%	F=F\left(T, V, \xi\right) \\
	%	G=G\left(T, p, \xi\right)
	%\end{gather}
	%\begin{gather}
	%	\dd{U}=T\dd{S}-p\dd{V}-A\dd{\xi} \\
	%	\dd{H}=T\dd{S}+V\dd{p}-A\dd{\xi} \\
	%	\dd{F}=-S\dd{T}-p\dd{V}-A\dd{\xi} \\
	%	\dd{G}=-S\dd{T}+V\dd{p}-A\dd{\xi}
	%\end{gather}
	%Pochodne cząstkowe:
	%\begin{table}[H]
	%\centering
	%\begin{tabular}{lll}
	%	\(\left(\pdv{U}{S}\right)_{V, \xi}=T\) & \(\left(\pdv{U}{V}\right)_{S, \xi}=-p\) & \(\left(\pdv{U}{\xi}\right)_{S, V}=-A\) \\
	%	\(\left(\pdv{H}{S}\right)_{p, \xi}=T\) & \(\left(\pdv{H}{p}\right)_{S, \xi}=V\) & \(\left(\pdv{H}{\xi}\right)_{S, p}=-A\) \\
	%	\(\left(\pdv{F}{T}\right)_{V, \xi}=-S\) & \(\left(\pdv{F}{V}\right)_{T, \xi}=-p\) & \(\left(\pdv{F}{\xi}\right)_{T, V}=-A\) \\
	%	\(\left(\pdv{G}{T}\right)_{p, \xi}=-S\) & \(\left(\pdv{G}{p}\right)_{T, \xi}=V\) & \(\left(\pdv{G}{\xi}\right)_{T, p}=-A\) \\
	%\end{tabular}
	%\end{table}
	%U, H, F, G są funkcjami stanu, więc pochodne mieszane są sobie równe:
	%\begin{table}[H]
	%\centering
	%\begin{tabular}{lll}
	%	\(\left(\pdv{T}{V}\right)_{S, \xi}=-\left(\pdv{p}{S}\right)_{V, \xi}\) & \(\left(\pdv{T}{\xi}\right)_{S, V}=-\left(\pdv{A}{S}\right)_{V, \xi}\) & \(\left(\pdv{p}{\xi}\right)_{S, V}=\left(\pdv{A}{V}\right)_{S, \xi}\) \\
	%	\(\left(\pdv{T}{p}\right)_{S, \xi}=\left(\pdv{V}{S}\right)_{p, \xi}\) & \(\left(\pdv{T}{\xi}\right)_{S, p}=-\left(\pdv{A}{S}\right)_{p, \xi}\) & \(\left(\pdv{V}{\xi}\right)_{S, p}=-\left(\pdv{A}{p}\right)_{S, \xi}\) \\
	%	\(\left(\pdv{S}{V}\right)_{T, \xi}=\left(\pdv{p}{T}\right)_{V, \xi}\) & \(\left(\pdv{S}{\xi}\right)_{T, V}=\left(\pdv{A}{T}\right)_{V, \xi}\) & \(\left(\pdv{p}{\xi}\right)_{T, V}=\left(\pdv{A}{V}\right)_{T, \xi}\) \\
	%	\(\left(\pdv{S}{p}\right)_{T, \xi}=-\left(\pdv{V}{T}\right)_{p, \xi}\) & \(\left(\pdv{S}{\xi}\right)_{T, p}=\left(\pdv{A}{T}\right)_{p, \xi}\) & \(\left(\pdv{V}{\xi}\right)_{T, p}=-\left(\pdv{A}{p}\right)_{T, \xi}\)
	%\end{tabular}
	%\end{table}
	%Pochodne po temperaturze:
	%\begin{table}[H]
	%\centering
	%\begin{tabular}{ll}
	%	\(\left(\pdv{U}{T}\right)_{V, \xi}=C_{V, \xi}\) & \(\left(\pdv{S}{T}\right)_{V, \xi}=\frac{C_{V, \xi}}{T}\) \\
	%	\(\left(\pdv{H}{T}\right)_{p, \xi}=C_{p, \xi}\) & \(\left(\pdv{S}{T}\right)_{p, \xi}=\frac{C_{p, \xi}}{T}\) \\
	%	\(\left(\pdv{F}{T}\right)_{V, \xi}=-S\) & \(\left(\pdv{A}{T}\right)_{V, \xi}=\Delta_{r}S_{V}\) \\
	%	\(\left(\pdv{G}{T}\right)_{p, \xi}=-S\) & \(\left(\pdv{A}{T}\right)_{p, \xi}=\Delta_{r}S_{p}\)
	%\end{tabular}
	%\end{table}
	%\begin{gather}
	%	A\dd{\xi}=T\dd{S}-\dd{U}-p\dd{V} \\
	%	A\dd{\xi}=T\dd{S}-\dd{H}+V\dd{p}
	%\end{gather}
	%Podstawiając do tego odpowiednio: 
	%\begin{gather}
	%\begin{split}
	%	\dd{S}=\left(\pdv{S}{T}\right)_{V, \xi}\dd{T}+\left(\pdv{S}{V}\right)_{T, \xi}\dd{V}+\left(\pdv{S}{\xi}\right)_{T, V}\dd{\xi} \\
	%	\dd{U}=\left(\pdv{U}{T}\right)_{V, \xi}\dd{T}+\left(\pdv{U}{V}\right)_{T, \xi}\dd{V}+\left(\pdv{U}{\xi}\right)_{T, V}\dd{\xi}
	%\end{split} \\
	%\begin{split}
	%	\dd{S}=\left(\pdv{S}{T}\right)_{p, \xi}\dd{T}+\left(\pdv{S}{p}\right)_{T, \xi}\dd{p}+\left(\pdv{S}{\xi}\right)_{T, p}\dd{\xi} \\
	%	\dd{H}=\left(\pdv{H}{T}\right)_{p, \xi}\dd{T}+\left(\pdv{H}{p}\right)_{T, \xi}\dd{p}+\left(\pdv{H}{\xi}\right)_{T, p}\dd{\xi}
	%\end{split}
	%\end{gather}
	%otrzymujemy:
	%\begin{multline}
	%	A\dd{\xi}=\left[T\left(\dv{S}{T}\right)_{V, \xi}-\left(\pdv{U}{T}\right)_{V, \xi}\right]\dd{T}+\left[T\left(\dv{S}{V}\right)_{T, \xi}-\left(\pdv{U}{V}\right)_{T, \xi}-p\right]\dd{V} \\
	%	+\left[T\left(\dv{S}{\xi}\right)_{T, V}-\left(\pdv{U}{\xi}\right)_{T, V}\right]\dd{\xi}
	%\end{multline}
	%\begin{multline}
	%	A\dd{\xi}=\left[T\left(\dv{S}{T}\right)_{p, \xi}-\left(\pdv{H}{T}\right)_{p, \xi}\right]\dd{T}+\left[T\left(\dv{S}{p}\right)_{T, \xi}-\left(\pdv{H}{p}\right)_{T, \xi}+V\right]\dd{p} \\
	%	+\left[T\left(\dv{S}{\xi}\right)_{T, p}-\left(\pdv{U}{\xi}\right)_{T, p}\right]\dd{\xi}
	%\end{multline}
	%Wyrażenia przy każdej różniczce powinny być sobie równe:
	%\begin{gather}
	%	T\left(\dv{S}{T}\right)_{V, \xi}-\left(\pdv{U}{T}\right)_{V, \xi}=0 \\
	%	T\left(\dv{S}{V}\right)_{T, \xi}-\left(\pdv{U}{V}\right)_{T, \xi}-p=0 \\
	%	A=T\left(\dv{S}{\xi}\right)_{T, V}-\left(\pdv{U}{\xi}\right)_{T, V} \\
	%	T\left(\dv{S}{T}\right)_{p, \xi}-\left(\pdv{H}{T}\right)_{p, \xi}=0 \\
	%	T\left(\dv{S}{p}\right)_{T, \xi}-\left(\pdv{H}{p}\right)_{T, \xi}+V=0 \\
	%	A=T\left(\dv{S}{\xi}\right)_{T, p}-\left(\pdv{U}{\xi}\right)_{T, p}
	%\end{gather}
	%Stosując wzory wynikające z równych pochodnych mieszanych możemy to przekształcic do: 
	%\begin{gather}
	%	\left(\pdv{U}{V}\right)_{T, \xi}=-p+T\left(\pdv{p}{T}\right)_{V, \xi} \\
	%	\left(\pdv{H}{p}\right)_{T, \xi}=V-T\left(\pdv{V}{T}\right)_{p, \xi}
	%\end{gather}
	%\pagebreak
	%\section{Dodatek C}
	%
	%\begin{description}
	%	\item[Energia wewnętrzna] U; suma wszytkich energii kinetycznych oraz potencjalnych cząstek składających się na ciało
	%	\item[Praca] W; energia dostarczona do układu poprzez siły makroskopowe
	%	\item[Ciepło] Q; energia dostarczona do układu poprzez oddziaływania inne niż praca
	%	\item[Entropia (termodynamika klasyczna)] S; funkcja stanu układu; wielkość opisująca samorzutność przemian w systemie; wprowadzona po raz pierwszy przez Clausiusa jako \(dS=\frac{\dbar Q^\circ}{T}\) \cite{orlik}
	%	\item[Entropia (termodynamika statystyczna)] S; wielkość opisująca ilość mikrostanów przypisanych danemu makrostanowi; wprowadzona przez Boltzmanna: \(S=k_B\ln{\Omega}\) \cite{landau}
	%	\item[0 Zasada Termodynamiki] Jeżeli układy A i C są w równowadze termodynamicznej oraz B i C to A i B są w równowadze termodynamicznej
	%	\item[I Zasada Termodynamiki] Zwiększenie energii wewnętrznej może nastąpić poprzez pracę lub ciepło; \(dU=\dbar Q+\dbar W\)
	%	\item[II Zasada Termodynamiki] w układach izolowanych następuje wzrost entropii lub pozostaje ona stała (poza wyjątkiem fluktuacji); entropia podukładu może maleć \cite{landau}
	%	\item[III Zasada Termodynamiki] entropia kryształów doskonałych dąży do zera, gdy temperatura dąży do zera
	%\end{description}
	%\subsection{Zarys historyczny}
	%Termodynamika równowagowa zajmuje się procesami, w których ignoruje się upływ czasu, a przemiana jest kwazistatyczna. Oznacza to, że każdy stan pośredni można traktować jako stan równowagi termodynamicznej. Model taki jest wystarczający do opisu większości procesów. Można więc powiedzieć, że termodynamikę równowagową interesuje stan początkowy oraz końcowy. \par
	%Termodynamika nierównowagowa jednak zajmuje się dokładnie tym, co się dzieje w trakcie rzeczywistej przemiany i jest ona konieczna do opisu reakcji oscylacyjnych. Pierwsze przesłanki o istnieniu takowych sięgają końca XIX wieku. Były to reakcje w układach heterogenicznych, jak na przykład pierścienie Lieseganga lub oscylacje prądu płynącego przez ogniwo galwaniczne. Wyjaśnienie tych zjawisk wymagało, aby układ byl heterogeniczny i było w zgodzie z entropią Boltzmanna, według której spontaniczna organizacja jest niemożliwa. \cite{orlik}\par
	%Pierwszy model teoretyczny został przedstawiony przez Alfreda Lotka \cite{lotka}. Przez długi czas uważano, że nie mogą one przedstawiać rzeczywistych reakcji, ponieważ łamią II Z.T. według Boltzmanna. Jednak w 1921r. pokazano w reakcji Bray'a-Liebhafky'ego, że reakcje oscylacyjne w układach homogenicznych są możliwę. Jest to reakcja rokładu nadtlenku wodoru katalizowana jodanem (V). Jeszcze większy wpływ na rozwój termodynamiki nierównowagowej w kinetyce chemicznej były reakcje Biełousowa-Żabotyńskiego. Pierwszą reakcją z tej grupy została zaobserwowanaw 1959 w mieszaninie bromianu (V) potasu, siarczanie (VI) ceru (IV), kwasu malonowego oraz kwasu cytrynowego w rozcieńczonym kwasie siarkowym (VI). Została ona odkryta jako nieorganiczny analog cyklu Krebsa \cite{belousov_hist}. Istnienie takich reakcji jest jednak niezgodne z oryginalną definicją entropii Boltzmanna. \par
	%Innym zjawiskiem wyjaśnionym dzięki rozwoju termodynamiki nierównowagowej jest życie i ewolucja. Niepoprawne użycie II Z.T. może doprowadzić do wniosku, że powstanie złożonej struktury z chaosu powinno być niemożliwe. Wyjaśnienie takie błędnie wykorzystuje to prawo ignorując fakt, że układy biologiczne jak i cała Ziemia nie są układami izolowanymi.
	%\subsection{Generacja entropii}
	%II Zasada Termodynamiki odnosi się do układów zamkniętych.
	%\begin{equation}\label{entropia_izolowany}
	%	dS_{uk}\geq 0
	%\end{equation}
	%Wynika z niego, że w układach izolowanych entropia zawsze rośnie, a zatem spontaniczne uporządkowanie stabilnych struktur nie jest możliwe. \par
	%Zmianę entropii układu można zapisać jako wynikającą z przepływu ciepła oraz samorzutnej generacji entropii: 
	%\begin{equation}
	%	dS_{uk}=d_{e}S+d_{i}S_{uk}
	%\end{equation}
	%Dla układu zamkniętego: \(d_{e}S=0\), stąd wykorzystując \eqref{entropia_izolowany}:
	%\begin{equation}
	%	dS_{uk}=d_{i}S_{uk}
	%\end{equation}
	%gdzie
	%\begin{description}
	%	\item[\(dS_{uk}\)] całkowita zmiana entropii układu
	%	\item[\(d_{e}S=\frac{\dbar Q}{T_{uk}}\)] zmiana wynikająca z przepływu ciepła
	%	\item[\(d_{i}S_{uk}\geq 0\)] zmiana wynikająca z produkcji entropii
	%\end{description}
	%Dla każdego układu produkcja entropii jest dodatnia. Wynika z tego argumentu także, że jest to prawdziłe dla każdego podukładu należącego do danego układu niezależnie od jego wielkości. Zasada ta ma więc charakter lokalny.
	%\section{Dodatek D}
	
	%\section{Dodatek E}
	%Uogólniona forma modelu bruskelator ma postać:
	%\begin{center}
	%	\ce{A <=>[k_{1}][k_{-1}] X} \\
	%	\ce{2X + Y <=>[k_{2}][k_{-2}] 3X} \\
	%	\ce{B + X <=>[k_{3}][k_{-3}] D + Y} \\
	%	\ce{X <=>[k_{4}][k_{-4}] E}.
	%\end{center}
	%Reakcje w nim są odwracalne i przebiegają przy różncyh stałych prędkości reakcji oznaczonych \(k_{i}\) oraz \(k_{-i}\) dla rekacji odpowiednio w prawą i lewą stronę. \par
	%Całkowita zmiana reagentów \(X\) oraz \(Y\) ma postać:
	%\begin{align}
	%	\dv{[X]}{t} &=& k_{1}[A] +& k_{2}[X]^{2}[Y] - k_{3}[B][X] - k_{4}[X] - k_{-1}[X] &-& k_{-2}[X]^{3} + k_{-3}[D][Y] + k_{-4}[E] \\
	%	\dv{[Y]}{t} &=& -& k_{2}[X]^{2}[Y] + k_{3}[B][X] &+& k_{-2}[X]^{3} - k_{-3}[D][Y]
	%\end{align}
	%Rozdzielam przyrosty na dwie części, odpowiadające reakcjom w prawą oraz lewą stronę:
	%\begin{align}
	%	\dv{[X]_{1}}{t} &= k_{1}[A] + k_{2}[X]^{2}[Y] - k_{3}[B][X] - k_{4}[X] \\
	%	\dv{[Y]_{1}}{t} &= - k_{2}[X]^{2}[Y] + k_{3}[B][X] \\
	%	\dv{[X]_{2}}{t} &= - k_{-1}[X] - k_{-2}[X]^{3} + k_{-3}[D][Y] + k_{-4}[E] \\
	%	\dv{[Y]_{2}}{t} &= + k_{-2}[X]^{3} - k_{-3}[D][Y]
	%\end{align}
	%Stany stacjonarne odpowiadające odpowiednio \([X]_{1}\) i \([Y]_{1}\) oraz \([X]_{2}\) i \([Y]_{2}\) to:
	%\begin{align}
	%	\dv{[X]_{st, 1}}{t} &= \frac{k_{1}[A]}{k_{4}}; & \dv{[Y]_{st, 1}}{t} &= \frac{k_{3}k_{4}[B]}{k_{1}k_{2}[A]} \\
	%	\dv{[X]_{st, 2}}{t} &= \frac{k_{-4}[E]}{k_{-1}}; & \dv{[Y]_{st, 2}}{t} &= \frac{k_{-2}k_{-4}^{3}[E]^{3}}{k_{-1}^{3}k_{-3}[D]}
	%\end{align}
	%Przyjmuję, że mogę dowolnie kontrolować stężenia reagentów \([A]\), \([B]\), \([D]\) i \([E]\). \([A]\) oraz \([B]\) pozostają dowolnymi parametrami, natomiast \([D]\) i \([E]\) są zależne od innych parametrów. Po przyrównaniu \([X]_{st, 1}\) oraz \([X]_{st, 2}\) i analogicznie dla \([Y]\) otrzymujemy wartości dla \([D]\) oraz \([E]\):
	%\begin{align}
	%	[D] &= \frac{k_{1}^{4}k_{2}k_{-2}[A]^{4}}{k_{3}k_{-3}k_{4}^{4}[B]} \\
	%	[E] &= \frac{k_{1}k_{-1}[A]}{k_{4}k_{-4}}
	%\end{align}
	%Wspólna wartość stężeń dla stanu stacjonarnego:
	%\begin{align}
	%	[X]_{st} &= \frac{k_{1}[A]}{k_{4}} \\
	%	[Y]_{st} &= \frac{k_{3}k_{4}[B]}{k_{1}k_{2}[A]}
	%\end{align}
	%Dla zwiększenia przejrzystości równań wprowadzam oznaczenia: 
	%\begin{align}
	%	[X] &= x[X]_{st} = x\frac{k_{1}[A]}{k_{4}} \\
	%	[Y] &= y[Y]_{st} = y\frac{k_{3}k_{4}[B]}{k_{1}k_{2}[A]} \\
	%	\tau &= k_{4}t \\
	%	a &= \frac{k_{3}[B]}{k_{4}} \\
	%	b &= \frac{k_{1}^{2}k_{2}[A]^{2}}{k_{4}^{3}} \\
	%	c &= \frac{k_{-1}}{k_{4}} \\
	%	d &= \frac{k_{1}^{4}k_{2}k_{-2}[A]^{4}}{k_{3}k_{4}^{5}[B]}
	%\end{align}
	%Równania różniczkowe mają wtedy postać
	%\begin{align}
	%	\dv{x}{\tau} &= 1 + ax^{2}y - ax - x - cx - bc^{3} + by + c \\
	%	\dv{y}{\tau} &= -bx^{2}y + bx + dx^{3} - dy, 
	%\end{align}
	%a punkt stacjonarny występuje dla \(x=1, y=1\). 
	%Po wprowadzeniu podstawienia:
	%\begin{gather*}
	%	\gamma = x - 1 \\
	%	\vartheta = y - 1
	%\end{gather*}
	%i linearyzacji otrzymujemy:
	%\begin{align}
	%	\dv{\gamma}{\tau} &= (a - c - 3b - 1)\gamma + (a + b)\vartheta \\
	%	\dv{\vartheta}{\tau} &= (- b + 3d)\gamma + (- b - d)\vartheta
	%\end{align}
	%
	%% do usuniecia?
	%Równanie charakterystyczne:
	%\begin{equation}
	%	\lambda^{2} - (a - c - 4b - d - 1)\lambda + (-4ad + bc + cd + 4b^{2} + b + d)
	%\end{equation}
	%Powinowactwo chemiczne w stanie równowagi każdego z równań z osobna wynosi \(A=0\) \cite{pigon1}. W ogólnej postaci ma ono postać:
	%\begin{equation}
	%	A = A_{0} - RT\ln(\prod_{i}c_{i}^{\nu_{i}}), 
	%\end{equation}
	%gdzie \(R\) to uniwersalna stała gazowa, \(T\) - temperatura bezwzględna, \(c_{i}\) - stężenie \(i\)-tego składnika, a \(\nu_{i}\) to współczynnik stechiometryczny reagenta \(i\) (dodatni dla produktów po prawej stronie, ujemny dla substratów po lewej). \(RT\) jest jedynie stałą i na potrzeby symulacji przyjąłem \(RT=1\). Otrzymujemy dla każdej z reakcji odpowiednio:
	%\begin{align}
	%	1: & \ln(\frac{1}{cx}) \\
	%	2: & \ln(\frac{by}{dx}) \\
	%	3: & \ln(\frac{bx}{dy}) \\
	%	4: & \ln(\frac{x}{c}).
	%\end{align}
	%Liczba postępu reakcji wyrażona jest równością: 
	%\begin{equation}
	%	\dd{\xi} = \frac{\dd{n_{i}}}{\nu_{i}}
	%\end{equation}
	%dla dowolnego reagenta, lub używając \(\dd{c_{i}} = \frac{\dd{n_{i}}}{V}\), gdzie \(V\) jest objętością, która także mogę przyjąc, że jest równa \(V=1\). Otrzymane liczby postępu reakcji dla poszczególnych reakcji:
	%\begin{align}
	%	\dv{\xi_{1}}{\tau} &= [X]_{st}(1 - cx) \\
	%	\dv{\xi_{2}}{\tau} &= [X]_{st}(ax^{2}y - \frac{ad}{b}x^{3}) \\
	%	\dv{\xi_{3}}{\tau} &= [X]_{st}(ac - \frac{ad}{b}y) \\
	%	\dv{\xi_{4}}{\tau} &= [X]_{st}(x - c).
	%\end{align}
	%\([X]_{st}\) można oczywiście przyjąć, że jest równe \([X]_{st}=1\). Z prawa de Dondera \(T\dd_{i}S=\sum_{r}A_{r}\xi_{r}\) przyjmując \(T=1\) otrzymujemy
	%\begin{equation}
	%	\dv{_{i}S}{\tau} = \ln(\frac{1}{cx})(1 - cx) + \ln(\frac{by}{dx})(ax^{2}y - \frac{ad}{b}x^{3}) + \ln(\frac{bx}{dy})(ac - \frac{ad}{b}y) + \ln(\frac{x}{c})(x - c)
	%\end{equation}
	%Wykresy otrzymane z przeprowadzonej symulacji dla kroku symulacji \(h = \dd{\tau} = 0,001\) oraz warunku początkowego \(x=1, y=2\):
	%\begin{figure}[H]
	%	\centering
	%	\includegraphics{"brusselator_rev1; a=9, b=1, c=1, d=0.1".png}
	%	\caption{Wykres fazowy dla a=9, b=1, c=1, d=0,1}
	%\end{figure}
	%\begin{figure}[H]
	%	\centering
	%	\includegraphics{"brusselator_rev2; a=9, b=1, c=1, d=0.1".png}
	%	\caption{Zależność wielkości \(x\) oraz \(y\) od \(\tau\)}
	%\end{figure}
	%\begin{figure}[H]
	%	\centering
	%	\includegraphics{"brusselator_rev3; a=9, b=1, c=1, d=0.1".png}
	%	\caption{Zależność wielkości \(S\) oraz \(\dv{S}{\tau}\) od \(\tau\)}
	%\end{figure}
	\section*{Dodatek C: Kod}\label{sec: dodatek C}
	\addcontentsline{toc}{section}{Dodatek C: Kod}
	Na potrzeby analizy modeli oscylacyjnych reakcji chemicznych napisałem w języku Python program używany do ich symulacji. 
	\begin{verbatim}
		import numpy as np
		import scipy as sci
		import matplotlib.pyplot as plt
		
		
		def vec_grid(function, xlim, ylim, xnodes, ynodes):
			"""
			Rysowanie pola wektorowego
			"""
		    x = np.linspace(xlim[0], xlim[1], xnodes)
		    y = np.linspace(ylim[0], ylim[1], ynodes)
		    X, Y = np.meshgrid(x, y)
		    U, V = function(0, [X, Y])[0: -1, :, :]
		    plt.quiver(X, Y, U / np.sqrt(U ** 2 + V ** 2), V / np.sqrt(U ** 2 + V ** 2), 
		    U ** 2 + V ** 2, angles='xy')
		
		
		def simulation(function, number_of_equations, length_of_sim, time_sample_rate, 
		init_val, color, xlim, ylim):
			"""
			Przeprowadzanie symulacji za pomocą schematów krokowych
			"""
		    time_step = 1 / time_sample_rate
		    time = np.linspace(0, length_of_sim, 
		    int(length_of_sim * time_sample_rate + 1))
		    value = np.empty((number_of_equations, 
		    int(length_of_sim * time_sample_rate + 1)))
		    initial_value = np.array(init_val)
		    value[:, 0] = initial_value
		    value[:, 1] = diff_eq_1(time, value, 0, time_step, function)
		    value[:, 2] = diff_eq_2(time, value, 1, time_step, function)
		    value[:, 3] = diff_eq_3(time, value, 2, time_step, function)
		
		    plt.rcParams.update({
		        "text.usetex": True,
		        "font.family": "sans-serif"
		    })
		
			# Rysowanie wykresów
		    for i in range(3, int(length_of_sim * time_sample_rate)):
		        value[:, i + 1] = diff_eq_4(time, value, i, time_step, function)
		    plt.plot(value[0, :], value[1, :], color=color, linestyle=' ', marker='.', 
		    label=r'$x=%.1f, y=%.1f$' % (initial_value[0], initial_value[1]))
		    plt.xlabel(r'$x$')
		    plt.ylabel(r'$y$')
		    vec_grid(function, xlim, ylim, 21, 21)
		    plt.legend()
		    plt.show()
		
		    plt.plot(time, value[0, :], color=color, linestyle='-', marker=None, 
		    label=r'$x=%.1f, y=%.1f: x$' % (initial_value[0], initial_value[1]))
		    plt.plot(time, value[1, :], color=color, linestyle='--', marker=None, 
		    label=r'$x=%.1f, y=%.1f: y$' % (initial_value[0], initial_value[1]))
		    plt.xlabel(r'$\tau$')
		    plt.ylabel(r'$x, y$')
		    plt.legend()
		    plt.show()
		
		    plt.plot(time, value[2, :], color=color, linestyle='-', marker=None, 
		    label=r'$S$')
		    plt.plot(time, function(time, value)[2], color=color, linestyle='--', 
		    marker=None, label=r'$\frac{dS}{d\tau}$')
		    plt.xlabel(r'$\tau$')
		    plt.ylabel(r'$S, \frac{dS}{d\tau}$')
		    plt.axhline(y=0, color='k')
		    plt.axvline(x=0, color='k')
		    plt.legend()
		    plt.show()


		def brusselator_mod(time, value):
			"""
			Model bruskelator z reakcjami w jedną stronę
			Zwraca wartość funkcji f(y) w dy/dx=f(y)
			"""
		    a = 7
		    b = 4
		    return np.array([1 + a * value[0] ** 2 * value[1] - (a + 1) * value[0],
		                     - b * value[0] ** 2 * value[1] + b * value[0],
		                     np.zeros_like(value[0])], dtype='float64')


		def brusselator_rev(time, value):
			"""
			Model bruskelator z reakcjami w dwie strony
			Zwraca wartość funkcji f(y) w dy/dx=f(y)
			"""
		    a = 9
		    b = 1
		    b_1 = b
		    b_2 = b
		    c = 1
		    d = 0.1
		    return np.array([1 + c - (a + c + 1) * value[0] + b_2 * value[1] + 
		                     a * value[0] ** 2 * value[1] - b_2 * value[0] ** 3,
		                     b_1 * value[0] - d * value[1] - b_1 * value[0] ** 2 * 
		                     value[1] + d * value[0] ** 3,
		                     np.log(1 / (c * value[0])) * (1 - c * value[0]) +
		                     np.log((b * value[1]) / (d * value[0])) * 
		                     (a * value[0] ** 2 * value[1] - a * d / b * value[0] ** 3) +
		                     np.log((b * value[0]) / (d * value[1])) * (a * c - 
		                     a * d / b * value[1]) +
		                     np.log(value[0] / c) * (value[0] - c)], dtype='float64')


		def lotka_mod(time, value):
			"""
			Model Lotki
			Zwraca wartość funkcji f(y) w dy/dx=f(y)
			"""
		    a = 0.1
		    return np.array([a - a * value[0] * value[1],
		                     value[0] * value[1] - value[1]], dtype='float64')


		def lotka_volterra_mod(time, value):
			"""
			Model Lotki-Volterry
			Zwraca wartość funkcji f(y) w dy/dx=f(y)
			"""
		    a = 1
		    return np.array([a * value[0] - a * value[0] * value[1],
		                     value[0] * value[1] - value[1]], dtype='float64')


		# Odpowiednio schematy 1, 2, 3, 5 w tabeli 1
		def diff_eq_1(time, value, i, h, function):
		    return value[:, i] + h * function(time[i], value[:, i])
		
		
		def diff_eq_2(time, value, i, h, function):
		    return value[:, i] + h * (3 * function(time[i], value[:, i])
		                              - function(time[i-1], value[:, i-1])) / 2
		
		
		def diff_eq_3(time, value, i, h, function):
		    return value[:, i] + h * (23 * function(time[i], value[:, i])
		                              - 16 * function(time[i-1], value[:, i-1])
		                              + 5 * function(time[i-2], value[:, i-2])) / 12
		
		
		def diff_eq_4(time, value, i, h, function):
		    return value[:, i] + h * (55 * function(time[i], value[:, i])
		                               - 59 * function(time[i-1], value[:, i-1])
		                               + 37 * function(time[i-2], value[:, i-2])
		                               - 9 * function(time[i-3], value[:, i-3])) / 24
		
		
		def diff_eq_5(time, value, i, h, function):
		    return value[:, i-3] + 4 * h * (2 * function(time[i], value[:, i])
		                                    - function(time[i-1], value[:, i-1])
		                                    + 2 * function(time[i-2], value[:, i-2])) / 3
		
		
		def main():
		    function = brusselator_rev
		    xlim = (0, 9)
		    ylim = (0, 3)
		    simulation(function, 3, 20, 1000, (1, 2, 0), 'r', xlim, ylim)


		if __name__ == "__main__":
		    main()

	\end{verbatim}
\end{document}